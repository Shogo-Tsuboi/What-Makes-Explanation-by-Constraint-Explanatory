\documentclass[dvipdfmx,twoside,11pt,uplatex]{jsarticle}

%ヘッダー
\pagestyle{myheadings}
\markboth{坪井------何が制約による説明を説明にしているのか?}{応用哲学会第15年次研究大会}
\setcounter{page}{1}

%maketitle情報
\title{何が制約による説明を説明にしているのか
\large{\\--- グラウンディングの観点から ---}
}
\author{坪井祥吾\thanks{一橋大学大学院社会学研究科,\texttt{tsuboishogo98@gmail.com}}}
\date{2023年4月22日}

\usepackage{modalops}

%原語併記
\newcommand{\myterm}[2]{{\emph{#1}}{[\emph{#2}]}}

%英字フォント
%\usepackage{ebgaramond}
\usepackage{mathpazo}  

%数式など
\usepackage{amsmath}
\usepackage{amsthm}
\usepackage{amssymb}
\usepackage{mathrsfs}
\usepackage{graphicx}

%url,ハイパーリンク
\usepackage{url}
\usepackage[setpagesize=false,dvipdfmx]{hyperref}

%文字化け
\usepackage{pxjahyper}

%引用スタイル
\usepackage{natbib}
\bibpunct[: ]{(}{)}{,}{a}{}{,}

%定理環境
\theoremstyle{definition}
\newtheorem{dfn}{Definition}
\newtheorem{thm}{Theorem}
\newtheorem{lem}{Lemma}
\newtheorem{fact}{Fact}
\newtheorem*{lem*}{}
\newtheorem{thesis}{テーゼ}
\newtheorem{thesis*}{}

\usepackage{mathrsfs}
\usepackage{bussproofs} %証明図
\usepackage{fancybox}
\usepackage{tikz}
\usetikzlibrary{graphs}
\usetikzlibrary{positioning}
\allowdisplaybreaks[1]

%圏点
\usepackage{pxrubrica}

%枠囲み
\usepackage{ascmac}
\usepackage{fancybox}

%tikz
\usepackage{tikz}
\usetikzlibrary{intersections, calc, arrows.meta}
\allowdisplaybreaks[1]

\begin{document}
\maketitle

\section{はじめに:科学的説明論とは何か}
なぜ恐竜は絶滅したのか?——巨大な小惑星がメキシコに衝突したからだ.なぜ2008年の大規模な不景気が生じたのか?——リーマン・ブラザーズが経営破綻したからだ.こうした説明を与えることは,科学という活動の重要な構成要素である.それゆえ,この種の説明は科学的説明[scientific explanation]と呼ばれ,科学哲学者の関心を集めてきた.そして科学哲学者は〈科学的説明とは何か?〉と問う.20世紀の終わり頃まで,この問いに対しては〈科学的説明とは因果的説明[causal explanation]のことである〉と答えるのが主流だった.つまり,ある出来事に科学的説明を与えるとは,その出来事の原因,ないしその出来事を成り立たせている因果メカニズムを特定することである,と.

しかし実際には,非因果的な種類の科学的説明が存在する(あるいは,少なくとも非因果的に見える説明が存在する)のである.この非因果的説明[non-causal explanation]は,近年の科学的説明論の一大トピックとなっている\footnote{
非因果的説明についてのサーベイとしては,\cite{kobayashietal2021noncausal,reutlinger2017ebc}などがある.
}.たとえば,次のような説明が非因果的説明の典型例である.〈ツルツルした水平の床の上に,赤ちゃんの乗ったベビーカーが置いてある.外から力が加わらないとして,赤ちゃんがいくらベビーカーの中で動き回っても,ベビーカーはほとんど動かなかった.なぜか?——運動量保存則が成り立つからだ.〉この説明では運動量保存則が説明項になっているが,しかし一般に法則は出来事の原因にはなりえない.それゆえ,この説明は非因果的だと言われるのである.

以上から,科学的説明には二つのサブカテゴリーがあることが分かる.因果的説明と非因果的説明である.そして,科学的説明\emph{論}とは,こうした種々の説明についての理論を立てることを目指す分野だと言える.別の言い方をすると,説明に対する説明,つまり\emph{メタ説明}を与えるのを目指すのが,科学的説明論である.詳細は\ref{howandwhat}節で見るが,メタ説明としては次の二種類を区別するべきだ,というのが本論のコアのアイデアである.つまり,「いかにして」の説明と,「何のおかげで」の説明の区別である.

さて,本論の狙いは,非因果的説明のさらなるサブカテゴリー,\emph{制約による説明}[explanation by constraint]についての\cite{Lange2011,Lange2013a,Lange2016,Lange2018b}の理論を批判的に検討することである.詳細は\ref{explanationbyconstraint}節で見るが,制約による説明とは,被説明項が必然性を備えており,その必然性が,説明項で述べられる制約から導かれるような説明のことである.先程のベビーカーの例も,Langeによれば制約による説明である.というのも,ベビーカーはたまたま動かなかったのではなく動きえなかったのであり,その必然性は,運動量保存則という制約に由来しているからである.Langeはこの制約による説明についてのメタ説明を与えているが,それに対する私の批判は,(1) そのメタ説明は「いかにして」の説明にはなっているが「何のおかげで」の説明にはなっていないし,(2) 説明の文脈依存性も汲み取れていない,という二点である.特に (1) について,「何のおかげで」の説明を行うためには\myterm{グラウンディング}{grounding}の概念を利用するのが望ましいと私は考えている.

以上を踏まえて,私が本論で擁護する理論を先に示してしまおう.
\begin{quote}
「$\phi_1, \phi_2, \ldots$は$\psi$を説明する」が非因果的説明であるのは,次の二つが成り立つ\emph{おかげ}である.
	\begin{enumerate}
	\item $\phi_1, \phi_2, \ldots$は$\psi$に関するグラウンディング情報$G$を含んでいる.	
	\item 「$\phi_1, \phi_2, \ldots$は$\psi$を説明する」の発話の文脈が$G$を要請している.
	\end{enumerate}
\end{quote}

\noindent(1) は\ref{firstargument}節で,(2) は\ref{secondargument}節で擁護する.

\section{表記法}
本論に入る前に,私が利用する表記法をまとめておく.まず文は,ギリシア文字$\phi, \psi, \ldots$で表す.そして\emph{説明}は,文の集合と文をつなぐ多対一のオペレーター$\xrightarrow{E}$として表現することにする(「E」は\underline{E}xplanationのEである).よって,$\phi_1, \phi_2, \ldots\xrightarrow{E}\psi$で,〈$\phi_1, \phi_2, \ldots$が$\psi$を説明する〉ということを意味する.

次に\emph{因果言明}は,やはり文の集合と文をつなぐ多対一のオペレーター$\xrightarrow{C}$として表現することにする(「C」は\underline{C}ausationのCである).よって,$\phi_1, \phi_2, \ldots\xrightarrow{C}\psi$で,〈$\phi_1, \phi_2, \ldots$が$\psi$を引き起こす〉ということを意味する.ただし,ここで因果言明は因果関係それ自体ではないことに注意せよ.あくまで因果言明は,しかじかの因果関係が成り立っていることを報告しているような言語的な表現である.よって,因果言明を文ベースで表すからといって,因果関係の関係項が文で支持されるような存在者(e.g., 事実,出来事)に限定されるということにはならない.

最後に\emph{グラウンディング言明}(グラウンディングの概念は\ref{firstargument}節で導入する)も,やはり文の集合と文をつなぐ多対一のオペレーター$\xrightarrow{G}$として表現することにする(「G」は\underline{G}roundingのGである).よって,$\phi_1, \phi_2, \ldots\xrightarrow{G}\psi$で,〈$\phi_1, \phi_2, \ldots$が$\psi$をグラウンドする〉ということを意味する.上と同様に,この表現の仕方はグラウンディング関係の関係項についての特定のコミットメントを含意しない.

\section{制約による説明(EBC)}\label{explanationbyconstraint}
制約による説明(\underline{E}xplanation \underline{B}y \underline{C}onstraint, 以下EBC)は,Langeが「発見」した,非因果的説明の一種である.そしてLange自身が,EBCとは何であるかについての学説を提示している.本節では,まずEBCの具体例とそこから観察される諸特徴を確かめ(\ref{examples}節),続いてLangeの見解を確かめる(\ref{langesview}節).

\subsection{制約による説明の例}\label{examples}
\cite{lange2016bwc,Lange2018bwcsum}は,EBCの具体例を豊富に挙げている.例えば,次のような説明がEBCである.
\begin{quote}
    \emph{いちご配り}\quad 母親が,3人の子供に23個のいちごを配ろうとしている.このとき,なぜ母親は,3人の子供に均等にいちごを配るのに必ず失敗するのか?------なぜなら,23は3で割り切れないからだ.
\end{quote}
さて,この説明は次のような構造を持つと言える.
\begin{description}
    \item[前提] 母親が,3人の子供に23個のいちごを配ろうとしている.
    \item[被説明項] なぜ母親は,3人の子供に均等にいちごを配るのに必ず失敗するのか?
    \item[説明項] 23は3で割り切れないから.
\end{description}
この説明は,次の三つの重要な特徴を持つ\citep[15--6]{Lange2018bwcsum}.
\begin{enumerate}
    \item \emph{非因果性},つまり,たしかに前提で因果的な状況に言及してはいるが,この説明の説明力[explanatory power]はその因果情報に由来するものではない,という特徴.母親が23個のいちごを持っていることや,子供が3人いることなどは因果的な状況だと言ってよいだろう.しかし,この説明の説明力はあくまで説明項の〈23は3で割り切れない〉に由来するものであり,それら因果的状況に由来するのではない.
    \item \emph{不可避性},つまり,被説明項は単に成り立っているのではなく,成り立たざるを得なかった,という特徴.(前提を固定した上で)何が起ころうと,母親はいちごを均等に配り切ることができなかったのである.
    \item \emph{説明項の強い必然性},つまり,説明項が因果を超越した必然性を持っている,という特徴.この例では,〈23は3では割り切れない〉という数学的真理が説明項となっている.この数学的真理は,世界の因果構造がどうあれ成り立つという意味で,因果を超越している.
\end{enumerate}

もしかすると,いちご配りの例は「科学的」説明には見えないかもしれない.というのも,あまりにも初等的な知識しか使っていないし,さらに被説明項が科学的探究の対象にならないように思われるからである.しかし,この説明が「科学的」に見えないことはそれほど重要ではない.というのも,いちご配り事例と同じ構造・特徴を持ちつつ,明白に「科学的」と言いうる説明の例も,Langeが提示しているからである.それは例えば次のようなものである.
\begin{quote}
    \emph{ベビーカー}\quad 水平の床の上に,ブレーキのかかっていないベビーカーが静かに置いてあり,さらにシートベルトをした赤ちゃんがそれに乗っているとする(前提).このとき,なぜ赤ちゃんがいくら動いても,ベビーカーはほとんど動きえないのか?(被説明項)------なぜなら,運動量保存則が成り立つからだ(説明項).
\end{quote}
この説明は「科学的」と言ってよいだろう.そして,いちご配りと同じ,前提-被説明項-説明項の三つ組構造を持つ.さらに,非因果性,不可避性,説明項の強い必然性の三つの特徴も持っている.(1) この説明の説明力は運動量保存則に由来しており,実際の因果構造には依存していないという点で,この説明は非因果的である.(2) ベビーカーが動かないということは不可避である.(3) 運動量保存則は,どのような因果関係が成り立つかを運動量保存則が支配している,という意味で,因果を超越した強さの必然性を持つ.

以上がEBCの事例と,それが持つ特徴である.この種の説明について,Lange自身が提示する学説を次に見る.

%制約による説明の特徴%\cite{Lange2013a}:
%\begin{itemize}
%    \item 数学的必然性に訴えていることがある.
%    \item 自然法則に訴えていない事が多い.
%    \item 自然法則に訴えている場合,その法則は比較的強い必然性を持っている.
%    \item 二つの現象が似ていることがcoincidenceでないことを説明できる(505--6).
%    \item 因果に言及しているっぽいケースでも,実はそうした因果構造はただ\emph{前提}されている(506).
%\end{itemize}

\subsection{Langeの見解}\label{langesview}
EBCについてのLangeの見解は,おおよそ次の三つのテーゼに要約できる.
\begin{enumerate}
\renewcommand{\labelenumi}{\Alph{enumi}.}
    \item EBCでは,説明項が\myterm{制約}{constraint}として働く.
    \item EBCの説明力は\emph{純粋に非因果的}である.
    \item 制約が持つ必然性は,\emph{必然性の階層構造}によって決まる.
\end{enumerate}

テーゼAは,EBCが\emph{どのように働くのか}についての特徴づけだと言える.このテーゼをLangeが支持していることは,例えば次の記述から裏付けられる.
\begin{quote}
    [運動量保存則]は,生じうる種々の力を制約[constrain]しているのである.それは,23が3で割り切れないことが,母親が3人の子供にいちごを配りうる仕方を制約しているのと全く同様である.\citep[17]{Lange2018bwcsum}
\end{quote}
この働き方は,因果的説明とEBCを区別することを動機づけるだろう.というのも,被説明項を\emph{生じさせた原因}を特定する,という仕方で因果的説明は働くのに対して,被説明項を\emph{支配する制約}を特定する,という(大きく異なるように見える)仕方で働くのがEBCだ,ということになるからである\footnote{
この対比は,ボトムアップの説明とトップダウンの説明の対比とよく似ている\citep{kitcher1985two}.ほげほげ.
}.

テーゼBは,EBCの説明力がどこから来るのか,換言すると,EBCは\emph{何のおかげで}説明になっているのか,についての特徴づけである.テキスト上の根拠は例えば次である.
\begin{quote}
    [いちご配り事例に関して,]母親が3人の子供と23個のいちごを持っていることは,この状況での彼女の失敗の原因ではなるが,しかし,この説明はその説明力を原因を特定するおかげで獲得しているわけではない.むしろ,母親の苺が子供たちに均等に配られなかったのは,[$\ldots$]それが\emph{ありえない}からなのである.\citep[15]{Lange2018bwcsum}\vspace{7pt}

    こうした説明\footnote{
    ただし,この引用部では,EBCではなく,「固有に数学的な説明[destinctively mathematical explanation]」が問題になっている.しかし,固有に数学的な説明はEBCの特別なサブクラスであり,さらに説明力が因果に由来しないという点でEBCと固有に数学的な説明は変わりない.
    }が非因果的だとされるのは,世界の因果関係のネットワークを記述することによってそれらが働いているわけではないからである.それらの説明力は別の仕方で生じている.たとえたまたま原因に訴えていたとしても,それらは原因としての原因に訴えているわけではない------つまり,それらは因果的パワーを利用しているわけではないのである.\citep[496]{lange2013dme}
\end{quote}
このテーゼが述べる,〈EBCの説明力の源泉が因果関係には存していない〉ということは,EBCと因果的説明の本質的な違いをしるしづけるという点で極めて重要である.というのも,因果的説明の説明力の源泉が因果にあり,EBCはそうでないとすると,それら二種類の説明は根本的に異なる種だと言えるからである.

テーゼCも,テーゼBと同様,EBCは何のおかげで説明になっているのかについての特徴づけである.テキスト上の根拠は以下である.
\begin{quote}
    制約による説明はその説明力を,被説明項の特別強い必然性がどこから来るのかについての情報を与えることによって獲得している.これは,因果的説明が,被説明項の因果来歴についての情報,ないし世界の因果関係のネットワークについての情報を与えることによって働くのと同様である.\citep[24]{Lange2018bwcsum}
\end{quote}
このテーゼについてはやや補足が必要だろう.このテーゼは,必然性についてのLangeの独特の見解に依拠している\citep{lange2009lawmakers}.その見解とは,(C-a) 必然性は階層構造をなし,そして (C-b) 必然性の各階層は反事実的条件文に基づいて決定される,というものである.

まず (C-a) の主張は,必然的真理には\myterm{反事実的な安定性}{counterfactual stability}がある,という観察から導かれる.ここで反事実的な安定性というのは,どのような反事実的な状況を想定しても,必然的真理は変わらず(安定して)成り立つ,という性質である.例えば,
\begin{align}
    たとえ私が今朝コーヒーを飲まなかったとしても,エネルギー保存則は成り立っていただろう.\label{1}
\end{align}
という反事実的条件文は真だろう.〈私が今朝コーヒーを飲まかった〉という反事実的な状況においても,エネルギー保存則という必然的真理は安定して成り立つのである.そしてLangeは,同様の観察を「より高いレベルで」繰り返す.例えば,
\begin{align}
    たとえエネルギー保存則が成り立たなかったとしても,23は3で割り切れなかっただろう.\label{2}
\end{align}
は真だろう.エネルギー保存則が成り立たないような反事実的な状況においても,〈23は3で割り切れない〉という数学的真理は安定して成り立つのである.ここで (\ref{1}) と (\ref{2}) の観察を合わせると,〈23は3で割り切れない〉という数学的真理は,エネルギー保存則という自然法則\emph{よりも強い}安定性を持つ,と言いたくなる.これが,必然性は階層構造をなす,というLangeの見解の背後にあるアイデアである.より一般化すると,数学的・論理的真理が最も強い必然性を持ち,基礎的な物理法則がそれに準ずる強さの必然性を持ち,$\cdots\cdots$という階層構造が成り立っているとされるのである.

(C-b) の主張も (C-a) と関連している.(C-b) の主張は,(C-a) での観察,つまり,必然的真理の種類によってどのような反事実的条件文が真になるかが決まる,という観察を,形而上学的に踏み込んで解釈するものである.すなわち,先の観察はあくまでどのような反事実的条件文が真になるのかという観察であったのだが,Langeはこれを必然性の各階層の個別化の基準とするのである.正確には,その基準は次のように述べられる\citep[29]{lange2009lawmakers}.
\begin{quote}
    (Stability)\quad 
    $\Gamma$を真理の集合とする.$\Gamma$は\emph{安定}である\\
    $\Longleftrightarrow$ $\Gamma$と整合的な全ての$p$について,全ての$m\in\Gamma$について,$p\necif m$.
\end{quote}
このように定義される真理の安定した集合が,必然性の階層構造の各層だということである.例えば,論理的真理全体の集合は安定している.論理的真理全体と整合的な(すなわち,全ての)任意の$p$について,任意の論理的真理$m$について,「もし$p$であっても$m$だっただろう」は成り立つだろうからである(これはまさに (\ref{2}) の例である).また,論理的真理プラス物理法則の集合も,同様にして安定であるように思われる.このようにして,上から数学的・論理的真理,物理法則,$\cdots\cdots$というように安定した真理の集合が定義されることになる.これが所望の必然性の階層構造だとLangeは主張するのである.重要なのは,(Stability)に現れる$\Leftrightarrow$は単なる実質同値ではなく,実在的な定義関係として意図されているということである.つまり,反事実は安定性よりも基礎的なのである.

EBCの事例と特徴,そしてEBCに関してLangeの支持する三つのテーゼについては以上である.改めて振り返ると,EBCについてのLangeの見解はごく単純化すると次のような構造を持っている.(矢印の曖昧さは次節で解消される.)
\begin{align*}
    反事実的条件法 \rightsquigarrow 必然性の階層 \rightsquigarrow 制約 \rightsquigarrow \text{EBC}
\end{align*}
すなわち,EBCの説明力は制約の説明力だとされ,制約は必然性の階層に適切な位置を持つのおかげでその説明力を持ち,そして必然性の階層は反事実的条件法によって個別化される,という見解である.従って,EBCの説明力の源泉は,究極的には反事実的条件法に由来する,ということになっている.

\section{「いかにして」の問いと「何のおかげで」の問い}\label{howandwhat}
前節では,EBCについてのLangeの見解を見た.次節移行で,その見解に対する反論を私は展開するが,本節ではその準備として,方法論的な検討を簡単に行う.

初心に戻ると,EBCを巡る議論は,EBCという種類の説明の本性を巡る議論だと言える.そして,一般に種類$E$の説明についての学説を提示しようとする際,私たちは次の二つの問いを区別するべきである.
    \begin{description}
    \item[「いかにして」の問い:]〈$\phi$が$\psi$を$E$の意味で説明する〉について,\emph{いかにして}$\phi$は$\psi$を説明しているのか?
    \item[「何のおかげで」の問い:]〈$\phi$が$\psi$を$E$の意味で説明する〉について,\emph{何のおかげで}$\phi$は$\psi$を説明できているのか?
    \end{description}
「いかにして」の問いは,\emph{問題の種類の説明が示す際立った性質にどのようなものがあるのか}を問うている.例えば因果的説明という種類の説明(e.g., 「窓が割れた.なぜならば,石を投げたからだ.」)について,「いかにして」の問いに対しては「$\phi$が$\psi$を産出する[produce]という仕方で,$\phi$は$\psi$を説明している」や,「$\phi$が$\psi$に違いをもたらす[make a difference]という仕方で,$\phi$は$\psi$を説明している」というように答えられうる.一方で「何のおかげで」の問いは,\emph{問題の種類の説明の説明力は何に由来しているのか}を問うている.再び因果的説明を例に取ると,「何のおかげで」の問いに対しては「$\phi$が$\psi$の原因であるおかげで,$\phi$は$\psi$を説明している」と答えられうる.

「いかにして」の問いと「何のおかげで」の問いは混同されるべきではない.この二つの問いの区別を把握するために,派生的性質/本質的性質の区別とのアナロジーを考えよう.例えば水は,沸点が100度である,4度で密度が最大になる,といった派生的な性質を持っている.これらの性質は,水の分子構造が$\text{H}_2\text{O}$であるという本質的性質(特に水素結合という結合の仕方))に由来している.これとアナロジカルに,任意の種類の説明は,ある派生的性質(「いかにして」の問いへの答え)と,それが由来するところの本質的性質(「何のおかげで」の問いへの答え)を持つ.再び因果的説明という種類を取り上げると,その派生的性質(e.g., 産出性,差異形成)は,本質的性質(因果)に由来するのである\footnote{
もちろん,この因果的説明についての見解は単なる一例である.産出性は派生的性質ではないとか,あるいは産出性こそが本質的性質である,といった見解が正しい可能性は残されている.重要なのは,そのような可能性においても,「いかにして」の問いと「何のおかげで」の問いの区別は維持されている,ということである.
}.

直観的には,「何のおかげで」の問いは「いかにして」の問いよりも根本的であると言いたくなる.というのも,「いかにして」の問いへの答えは「何のおかげで」の問いへの答えに依存しているように見えるからだ.しかし少なくとも本論ではそこまでの主張を行う必要はない.重要なのは,問題の種類の説明についての学説を満足な形で提示するためには,\emph{どちらの}問いに対しても答えを与えなければならない,ということである.水の研究において,派生的性質と本質的性質のいずれの研究も重要であることは明らかだと思う.同様の態度を,説明論にも向けるべきである.

さて,以上の方法論的な検討を踏まえて,Langeの見解に戻ろう.EBCという種類の説明について,Langeは「いかにして」の問いと「何のおかげで」の問いにどのようにして答えているだろうか? 三つのテーゼについて順に確認する.まず,LangeのテーゼAは,「いかにして」の問いへの答えだと解釈するのが穏当だろう.つまり,EBCでは,説明項が被説明項を制約するという\emph{仕方で}説明がなされるのである.続いて,テーゼBは,「何のおかげで」の問いへの答えとして意図されているように思われる.実際にLangeも,「おかげで[in virtue of]」や「なぜならば[because]」といった,「何のおかげで」の問いへの答えであることを示唆する表現を用いている.ただし,テーゼBはネガティブなテーゼである.つまり,EBCでは,説明力は因果\emph{に由来していない}ということである.最後のテーゼCも,「何のおかげで」の問いへの答えとして意図されているように思われる.さらに,これはポジティブなテーゼである.つまり,EBCでは,説明力は必然性の階層構造\emph{に由来する}(そして必然性の階層は反事実的条件法に由来する)とされるのである.よって,\ref{explanationbyconstraint}節の最後で示したLangeの見解の構造における矢印の曖昧性は,次のように解消される.ただし,$A\overset{I}{\rightsquigarrow}B$は$A$が$B$の「何のおかげで」の答えであること,$A\overset{H}{\rightsquigarrow}B$は$A$が$B$の「いかにして」の答えであることを意味する.
\begin{align*}
    反事実的条件法 \overset{I}{\rightsquigarrow} 必然性の階層 \overset{I}{\rightsquigarrow} 制約 \overset{H}{\rightsquigarrow} \text{EBC}
\end{align*}

このように理解されるLangeの見解に対して,以下,私は反論を二つ提示する.一つは,そもそもLangeは「何のおかげで」の問いに答えられていないのではないか,という反論である.もう一つは,仮に「何のおかげで」の問いには答えられているとして,Langeの見解は形而上学的に重すぎるコミットメントを持ってしまっている,という反論である.一つ目の反論は\ref{firstobjection}節で,二つ目の反論は\ref{secondobjection}節で行う.

\section{反論その1: 「何のおかげで」の問いに答えられていない}\label{firstobjection}

Langeは,EBCは因果的ではない,とか,EBCは因果的情報をその説明力の源泉としてない,とかいったネガティブなことはさかんに言っている\citep{Lange2013a}が,では何なのかをそもそもあまり言っていない.

様相的事実は説明されるべき事柄のはずだが,Langeはむしろそれを説明項にしている.

\section{反論その2:形而上学的に重すぎる}\label{secondobjection}

いかなる形而上学理論も,少なくとも次の二つの基準を満たしている程度に応じて評価される.
\begin{quote}
    \emph{ミニマル性}\quad 理論が措定する形而上学的存在者が(種的にないし数的に)少ないこと.
\end{quote}

\begin{quote}
    \emph{非実質性}\quad 理論が排除する他の形而上学的見解が少ないこと.
\end{quote}
これらの基準に照らすと,Langeの見解はミニマルでもないし,非実質的でもない.

ミニマル性については,

非実質性については,

\section{グラウンディングによるEBCの理論}

\subsection{(ミニマルな)グラウンディングの概念}

\subsection{グラウンドによる説明としての制約による説明}
$\phi\xrightarrow{e}\psi$が非因果的説明であるのは,次のことが成り立つ\emph{おかげ}である.
	\begin{enumerate}
	\item $\phi$は$\psi$に関するグラウンディング情報$G$を含んでいる.
	\end{enumerate}

\subsection{グラウンディングによる理論の一般性}


\section{おわりに}

\section{メモ}
\cite{Lange2009a,Lange2009b,Lange2010,Lange2011,Lange2012,Lange2013a,Lange2013b,Lange2013c,Lange2014,Lange2015,Lange2016,Lange2018a,Lange2018b}

\subsection{Laws and Lawmakers}
偶然[accident]と法則[law]の違い\citep[13]{Lange2009lawmakers}:
\begin{itemize}
    \item 法則を破る反事実的条件法の前件は,何らかの法則と不整合.
    \item 偶然を破る反事実的条件法の前件は,どの法則とも整合的でありうる.
\end{itemize}

Langeの法則観の最もコアにある原理,Nomic Preservationも出てくる\citep[13]{Lange2009lawmakers}:
\begin{quote}
    NP\quad $m$は法則である iff\\
    全ての法則と論理的に整合的であるような任意の反事実的な前提$p$のもとで,$m$は依然として成り立つだろう.
\end{quote}
ただしNPは修正が必要.
\begin{enumerate}
    \item 関連性問題(14).
    \item 文脈問題(14--5).
    \item 法則と論理的真理の区別問題(15--6).
    \item taken together問題(16--7).
    \item sub-nomic問題(17--20).
\end{enumerate}
これら全部に対処すると,こうなる\citep[20]{Lange2009lawmakers}:
\begin{quote}
    NP\quad $m$は法則である iff \\
    任意の文脈で,任意の$p$について,法則の集合$N$が存在して,\\
    $p$は全ての$n\in N$とあわせて論理的に整合的であり,かつ,$p\Box\rightarrow m$.
\end{quote}

さらにもう少し修正が必要.
\begin{itemize}
    \item 
\end{itemize}

Langeは,NPが法則の\emph{外延}を捉えてはいるかもしれないが,\emph{説明}はできてないのではないか,という点にちゃんと気づいている\citep[Sect.1,6]{Lange2009lawmakers}

\emph{sub-nomic stability}の定義\citep[29]{Lange2009lawmakers}:
\begin{dfn}
    $\Gamma$: sub-nomicな集合.\\
    $\Gamma$はsub-nomically stable\\
    $\Longleftrightarrow$ すべての文脈で,\\
    $\Gamma$と整合する全てのsub-nomicな$p$について,\\
    すべての$m\in\Gamma$について,\\
    $p\Box\rightarrow m$
\end{dfn}
こう定義した上で,\emph{法則全体からなる集合}$\Lambda$を,\emph{極大でない,最大のsub-nomicな集合}と定義する.こうすると,循環性の問題は回避される.

プライオリティ問題への言及もあり!!
\begin{quote}
    \emph{プライオリティ問題:}法則より反事実のほうが基礎的,というのはおかしいのでは?
\end{quote}
\citet[31]{Lange2009lawmakers}は,そもそもどっちが基礎的かという話はしていない.
\begin{quote}
    この章では,法則と反事実の間の特別な関係を\emph{同定}することにかかわっているだけで,この関係が\emph{なぜ}成り立つかを理解することにかかわっているわけではない.つまり,法則が反事実に責任を負っているか,反事実が法則に責任を負っているか,あるいはそれらのどちらでもないか,ということにはかかわっていない.こうした問題には2章と4章で取り組む.(31)
\end{quote}

\begin{quote}
    自然法則とは何であるかについてのいかなる形而上学的学説も,法則が複数の層をなす可能性の余地を残すべきである.\citep[41]{Lange2009lawmakers}
\end{quote}

\subsection{\cite{Lange2011}. Conservation Laws in Scientific Explanations}
BWTのCh.2の前半部分の元ネタ(たぶん).

単なる偶然[coincidence]と制約[constraint]の違いの話をしている論文.特に,エネルギー保存則が,\emph{もし制約だったとしたら}それを使った説明の説明力の源泉は何か,を特定しようとしている.(それが実際に制約\emph{である},ということを示そうという論文ではない.)

%一般に,ある法則的言明には,\emph{ボトムアップの}説明と\emph{トップダウンの}説明が与えられうる.E.g., アルキメデスの原理には,力学的メカニズムに訴えるボトムアップの説明と,エネルギー保存則に訴えるトップダウンの説明が与えられる.

制約と偶然は,それぞれ次のように反事実によって定義される.
\begin{dfn}
エネルギー保存則が制約である\\
$\Longleftrightarrow$
現実にはない追加の力が存在する$\Box\hspace{-4pt}\rightarrow$エネルギーは保存される.\\
($A\Box\hspace{-4pt}\rightarrow C$)
\end{dfn}
\begin{dfn}
エネルギー保存則が偶然である\\
$\Longleftrightarrow$
現実にはない追加の力が存在する$\Diamond\hspace{-4pt}\rightarrow$エネルギーは保存されない.\\
($A\Diamond\hspace{-4pt}\rightarrow\lnot C\equiv \lnot(A\Box\hspace{-4pt}\rightarrow C)$)
\end{dfn}
制約の定義1で,反事実的条件法の前件に法則が入ってくることから,制約は\emph{メタ法則}だと言える\citep[170]{Lange2012}.

\subsection{\cite{Lange2012}. There Sweep}
BWTのCh.2の後半部分の元ネタ.

エネルギー保存則が制約なのか偶然なのかという区別は,何が何よりも説明的に優先するか[explanatorily prior]の区別\citep[157]{Lange2012}.
\begin{itemize}
	\item 制約であるなら,エネルギー保存則$<$個々の領域でのエネルギー保存.
	\item 偶然であるなら,個々の領域でのエネルギー保存$<$エネルギー保存則.
\end{itemize}

\emph{法則}の定義(暫定版)\citep[171--2]{Lange2012}:
\begin{dfn}
mが法則\\
$\Longleftrightarrow$ 全ての法則と論理的に整合的などんな命題pについても,($p\Box\hspace{-4pt}\rightarrow m$).
\end{dfn}


\emph{sub-nomic}と\emph{nomic}の区別\citep[171]{Lange2012}:
\begin{dfn}
pがsub-nomic\\
$\Longleftrightarrow$ pの真理メーカーが法則でない.
\end{dfn}
\begin{dfn}
pがnomic\\
$\Longleftrightarrow$ pの真理メーカーが法則.
\end{dfn}

\emph{sub-nomicな安定性}の定義\citep[172]{Lange2012}:
\begin{dfn}
$\Gamma$: sub-nomicな命題の集合で,$\Gamma=Cn\Gamma$.\\
$\Gamma$がsub-nomicな安定性を持つ\\
$\Longleftrightarrow(\forall m\in\Gamma)(\forall p)(\Gamma\cup\{p\}$は整合的$\rightarrow(\lnot(p\Diamond\hspace{-4pt}\rightarrow\lnot m))$.
\end{dfn}

\emph{法則}の定義(修正版)\citep[173]{Lange2012}:
\begin{dfn}
mが法則\\
$\Longleftrightarrow$ (極大でない)sub-nomicallyに安定した集合にmが属する.
\end{dfn}

\begin{thm}
sub-nomicallyに安定した集合のクラスは階層的である.\citep[173]{Lange2012}
\end{thm}

Langeの見解の利点\citep[174]{Lange2012}:
\begin{itemize}
	\item 様々なレベルの制約を認められる.
	\item 高階の法則としての制約の役割をうまく説明できる.(上の層のやつが下の層のやつを制約する.)
	\item 制約が種々の可能性を特徴づけるメカニズムの説明にもなってる.
\end{itemize}

\emph{nomicな安定性}の定義\citep[175]{Lange2012}:
\begin{dfn}
$\Gamma$: nomicないしsub-nomicな真理の集合で,$\Gamma=Cn\Gamma$.\\
$\Gamma$がnomicな安定性を持つ\\
$\Longleftrightarrow(\forall m\in\Gamma)(\forall p)(\Gamma\cup\{p\}は整合的\rightarrow (\lnot(p\Diamond\hspace{-4pt}\rightarrow\lnot m))$
\end{dfn}

最後の節で,本質主義的な法則観を退けている.(それは制約/偶然の区別をつけられないので.)

\subsection{\cite{Lange2013a}. What Makes a Scientific Explanation Distinctively Mathematical?}
BWCのCh.1の元ネタ.

非因果的説明を(ざっくり)定義している\citep[487,491]{Lange2013a}.
\begin{dfn}
    説明タイプEが非因果的である\\
    $\Longleftrightarrow$Eは,被説明項が生じるのがいかにして\emph{不可避}である(しかも,因果的必然性よりも強い意味で)のかを示すことによって説明している.
\end{dfn}
因果的説明も定義してる\citep[493]{Lange2013a}.
\begin{dfn}
    説明タイプEが因果的である\\
    $\Longleftrightarrow$ Eは,世界の因果情報を与えることによって説明している.
\end{dfn}

DMEが何\emph{ではない}か:
\begin{quote}
非因果的説明が非因果的だとされるのは,それが,その説明力を,世界の因果構造を記述していることから引き出しているわけではないからである\citep[495]{Lange2013a}.

DMEが被説明項の原因をたまたま特定していることがあることは,それが被説明項の原因を記述すること\emph{のおかげで}働いていることを意味しない.母親が三人の子供を持っていることがいちご配りの失敗を説明するのは,それが彼女の失敗の原因である\emph{おかげ}ではなく,彼女の成功を数学的に不可能にしている\emph{おかげ}である\citep[496]{Lange2013a}.

[DME]が原因と被説明項との間にinvokeするいかなるつながりも,普通の偶然的な自然法則の\emph{おかげで}成り立っているわけではなく,特に数学的必然性の\emph{おかげで}成り立っているのである.\citep[496--7]{Lange2013a}
\end{quote}

DMEが何\emph{であるか},あるいは,いかにして働いているか:
\begin{quote}
    科学におけるDMEに現れる自然法則は,存在する特定の種類の原因を記述するような法則を\emph{超越}していなければならない,と私は提案する.\citep[505]{Lange2013a}

    説明項となる事実が説明をするのにふさわしいものとなっているのは,それが通常の因果法則よりも様相的により必然的である(ニュートンの第二法則や数学的事実のように)こと\emph{のおかげ}である.あるいは,問題となっている物理的なタスクや配置を構成するwhyの問いのコンテクストにおいて理解されているおかげである.
    \begin{itemize}
        \item これは一つ目が法則が説明項に来るようなEBC(つまり,タイプcのEBC)? そして二つ目が,因果的な条件が説明項に来るようなEBC(つまりnタイプのEBC)?
    \end{itemize}
\end{quote}

\subsection{\cite{Lange2018b}.Because Without Cause}
Langeの「制約による説明」についての概説的な論文.

ほげーー

\bibliographystyle{apa}
\bibliography{bibliography}

\end{document}  