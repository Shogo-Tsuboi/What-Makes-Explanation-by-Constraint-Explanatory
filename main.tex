\documentclass[dvipdfmx,twoside,11pt,uplatex]{jsarticle}

%ヘッダー
\pagestyle{myheadings}
\markboth{坪井------何が制約による説明を説明にしているのか?}{応用哲学会第15年次研究大会}
\setcounter{page}{1}

%maketitle情報
\title{何が制約による説明を説明にしているのか
\large{\\--- グラウンディングの観点から ---}
}
\author{坪井祥吾\thanks{一橋大学大学院社会学研究科,\texttt{tsuboishogo98@gmail.com}}}
\date{2023年4月22日}

%原語併記
\newcommand{\myterm}[2]{{\emph{#1}}{[\emph{#2}]}}

%英字フォント
%\usepackage{ebgaramond}
\usepackage{mathpazo}  

%数式など
\usepackage{amsmath}
\usepackage{amsthm}
\usepackage{amssymb}
\usepackage{mathrsfs}
\usepackage{graphicx}
\usepackage{modalops} %counterfactuals等

%url,ハイパーリンク
\usepackage{url}
\usepackage[setpagesize=false,dvipdfmx]{hyperref}

%文字化け
\usepackage{pxjahyper}

%コメントアウト
\usepackage{comment}

%引用スタイル
\usepackage{natbib}
\bibpunct[: ]{(}{)}{,}{a}{}{,}

%定理環境
\theoremstyle{definition}
\newtheorem{dfn}{定義}
\newtheorem{thm}{定理}
\newtheorem{lem}{補題}
\newtheorem{fact}{事実}
\newtheorem{prop}{命題}

\usepackage{mathrsfs}
\usepackage{bussproofs} %証明図
\usepackage{fancybox}
\usepackage{tikz}
\usetikzlibrary{graphs}
\usetikzlibrary{positioning}
\allowdisplaybreaks[1]

%圏点
\usepackage{pxrubrica}

%枠囲み
\usepackage{ascmac}
\usepackage{fancybox}

%tikz
\usepackage{tikz}
\usetikzlibrary{intersections, calc, arrows.meta}
\allowdisplaybreaks[1]

%例文
\usepackage{gb4e}

\begin{document}
\maketitle

\section{はじめに:科学的説明論とは何か}
なぜ恐竜は絶滅したのか?——巨大な小惑星がメキシコに衝突したからだ.なぜ2008年の大規模な不景気が生じたのか?——リーマン・ブラザーズが経営破綻したからだ.こうした説明を与えることは,科学という活動の重要な構成要素である.それゆえ,この種の説明は科学的説明[scientific explanation]と呼ばれ,科学哲学者の関心を集めてきた.そして科学哲学者は〈科学的説明とは何か?〉と問う.20世紀の終わり頃まで,この問いに対しては〈科学的説明とは因果的説明[causal explanation]のことである〉と答えるのが主流だった.つまり,ある出来事に科学的説明を与えるとは,その出来事の原因,ないしその出来事を成り立たせている因果メカニズムを特定することである,と.

しかし実際には,非因果的な種類の科学的説明が存在する(あるいは,少なくとも非因果的に見える説明が存在する)のである.この非因果的説明[non-causal explanation]は,近年の科学的説明論の一大トピックとなっている\footnote{
非因果的説明についてのサーベイとしては,\cite{kobayashietal2021noncausal,reutlinger2017ebc}などがある.
}.たとえば,次のような説明が非因果的説明の典型例である.〈ツルツルした水平の床の上に,赤ちゃんの乗ったベビーカーが置いてある.外から力が加わらないとして,赤ちゃんがいくらベビーカーの中で動き回っても,ベビーカーはほとんど動かなかった.なぜか?——運動量保存則が成り立つからだ.〉この説明では運動量保存則が説明項になっているが,しかし一般に法則は出来事の原因にはなりえない.それゆえ,この説明は非因果的だと言われるのである.

以上から,科学的説明には二つのサブカテゴリーがあることが分かる.因果的説明と非因果的説明である.そして,科学的説明\emph{論}とは,こうした種々の説明についての理論を立てることを目指す分野だと言える.別の言い方をすると,説明に対する説明,つまり\emph{メタ説明}を与えるのを目指すのが,科学的説明論である.詳細は\ref{howandwhat}節で見るが,メタ説明としては次の二種類を区別するべきだ,というのが本論のコアのアイデアである.つまり,「いかにして」の説明と,「何のおかげで」の説明の区別である.

さて,本論の狙いは,非因果的説明のさらなるサブカテゴリー,\emph{制約による説明}[explanation by constraint]についての\cite{Lange2011,Lange2013a,Lange2016,Lange2018b}の理論を批判的に検討することである.詳細は\ref{explanationbyconstraint}節で見るが,制約による説明とは,被説明項が必然性を備えており,その必然性が,説明項で述べられる制約から導かれるような説明のことである.先程のベビーカーの例も,Langeによれば制約による説明である.というのも,ベビーカーはたまたま動かなかったのではなく動きえなかったのであり,その必然性は,運動量保存則という制約に由来しているからである.Langeはこの制約による説明についてのメタ説明を与えているが,それに対する私の批判は,(1) そのメタ説明は「いかにして」の説明にはなっているが「何のおかげで」の説明にはなっていないし,(2) 説明の文脈依存性も汲み取れていない,という二点である.特に (1) について,「何のおかげで」の説明を行うためには\myterm{グラウンディング}{grounding}の概念を利用するのが望ましいと私は考えている.

以上を踏まえて,私が本論で擁護する理論を先に示してしまおう.
\begin{quote}
「$\phi_1, \phi_2, \ldots$は$\psi$を説明する」が非因果的説明であるのは,次の二つが成り立つ\emph{おかげ}である.
	\begin{enumerate}
	\item $\phi_1, \phi_2, \ldots$は$\psi$に関するグラウンディング情報$G$を含んでいる.	
	\item 「$\phi_1, \phi_2, \ldots$は$\psi$を説明する」の発話の文脈が$G$を要請している.
	\end{enumerate}
\end{quote}

\noindent(1) は\ref{firstargument}節で,(2) は\ref{secondargument}節で擁護する.

\section{制約による説明(EBC)}\label{explanationbyconstraint}
制約による説明(\underline{E}xplanation \underline{B}y \underline{C}onstraint, 以下EBC)は,Langeが「発見」した,非因果的説明の一種である.そしてLange自身が,EBCとは何であるかについての学説を提示している.本節では,まずEBCの具体例とそこから観察される諸特徴を確かめ(\ref{examples}節),続いてLangeの見解を確かめる(\ref{langesview}節).

\subsection{制約による説明の例}\label{examples}
\cite{lange2016bwc,Lange2018bwcsum}は,EBCの具体例を豊富に挙げている.例えば,次のような説明がEBCである.
\begin{quote}
    \emph{いちご配り}\quad 母親が,3人の子供に23個のいちごを配ろうとしている.このとき,なぜ母親は,3人の子供に均等にいちごを配るのに必ず失敗するのか?------なぜなら,23は3で割り切れないからだ.
\end{quote}
さて,この説明は次のような構造を持つと言える.
\begin{description}
    \item[前提] 母親が,3人の子供に23個のいちごを配ろうとしている.
    \item[被説明項] なぜ母親は,3人の子供に均等にいちごを配るのに必ず失敗するのか?
    \item[説明項] 23は3で割り切れないから.
\end{description}
この説明は,次の三つの重要な特徴を持つ\citep[15--6]{Lange2018bwcsum}.
\begin{enumerate}
    \item \emph{非因果性},つまり,たしかに前提で因果的な状況に言及してはいるが,この説明の説明力[explanatory power]はその因果情報に由来するものではない,という特徴.母親が23個のいちごを持っていることや,子供が3人いることなどは因果的な状況だと言ってよいだろう.しかし,この説明の説明力はあくまで説明項の〈23は3で割り切れない〉に由来するものであり,それら因果的状況に由来するのではない.
    \item \emph{不可避性},つまり,被説明項は単に成り立っているのではなく,成り立たざるを得なかった,という特徴.(前提を固定した上で)何が起ころうと,母親はいちごを均等に配り切ることができなかったのである.
    \item \emph{説明項の強い必然性},つまり,説明項が因果を超越した必然性を持っている,という特徴.この例では,〈23は3では割り切れない〉という数学的真理が説明項となっている.この数学的真理は,世界の因果構造がどうあれ成り立つという意味で,因果を超越している.
\end{enumerate}

もしかすると,いちご配りの例は「科学的」説明には見えないかもしれない.というのも,あまりにも初等的な知識しか使っていないし,さらに被説明項が科学的探究の対象にならないように思われるからである.しかし,この説明が「科学的」に見えないことはそれほど重要ではない.というのも,いちご配り事例と同じ構造・特徴を持ちつつ,明白に「科学的」と言いうる説明の例も,Langeが提示しているからである.それは例えば次のようなものである.
\begin{quote}
    \emph{ベビーカー}\quad 水平の床の上に,ブレーキのかかっていないベビーカーが静かに置いてあり,さらにシートベルトをした赤ちゃんがそれに乗っているとする(前提).このとき,なぜ赤ちゃんがいくら動いても,ベビーカーはほとんど動きえないのか?(被説明項)------なぜなら,運動量保存則が成り立つからだ(説明項).
\end{quote}
この説明は「科学的」と言ってよいだろう.そして,いちご配りと同じ,前提-被説明項-説明項の三つ組構造を持つ.さらに,非因果性,不可避性,説明項の強い必然性の三つの特徴も持っている.(1) この説明の説明力は運動量保存則に由来しており,実際の因果構造には依存していないという点で,この説明は非因果的である.(2) ベビーカーが動かないということは不可避である.(3) 運動量保存則は,どのような因果関係が成り立つかを運動量保存則が支配している,という意味で,因果を超越した強さの必然性を持つ.

以上がEBCの事例と,それが持つ特徴である.この種の説明について,Lange自身が提示する学説を次に見る.

%制約による説明の特徴%\cite{Lange2013a}:
%\begin{itemize}
%    \item 数学的必然性に訴えていることがある.
%    \item 自然法則に訴えていない事が多い.
%    \item 自然法則に訴えている場合,その法則は比較的強い必然性を持っている.
%    \item 二つの現象が似ていることがcoincidenceでないことを説明できる(505--6).
%    \item 因果に言及しているっぽいケースでも,実はそうした因果構造はただ\emph{前提}されている(506).
%\end{itemize}

\subsection{Langeの見解}\label{langesview}
EBCについてのLangeの見解は,おおよそ次の三つのテーゼに要約できる.
\begin{enumerate}
\renewcommand{\labelenumi}{\Alph{enumi}.}
    \item EBCでは,説明項が\myterm{制約}{constraint}を記述しており,それが被説明項で記述される出来事の\myterm{不可避性}{inevitability}を導いている.
%    \item 制約が持つ説明力は,それが\emph{比較的強い必然性}を持つことに由来する.
    \item 説明項で記述される制約は,被説明項で記述される出来事の不可避性------すなわち,\emph{比較的強い必然性}------が何に由来するのかを特定している.
    \item 必然性の強さは,いかなる\myterm{仮定法の事実}{subjunctive fact}が成り立っているかによって決定される.
\end{enumerate}
テーゼAは\cite{lange2011const,lange2012sweep,lange2013dme,Lange2018bwcsum}の一連の仕事において擁護されているものであり,テーゼCはその前に著された\cite{lange2009lawmakers}で展開されていて,テーゼBはこのテーゼCをEBCに応用したものである.以下,順番にその内実を確認しよう.

\subsubsection{テーゼA}\label{a}
テーゼAは,EBCが\emph{どのように働くのか}についての特徴づけだと言える.まず,EBCの説明項はある種の\emph{制約}であるとされる.例えば,ベビーカー事例といちご配り事例を例に挙げて,Langeは次のように述べる.
\begin{quote}
    [運動量保存則]は,生じうる種々の力を制約[constrain]しているのである.それは,23が3で割り切れないことが,母親が3人の子供にいちごを配りうる仕方を制約しているのと全く同様である.\citep[17]{Lange2018bwcsum}
\end{quote}
すなわち,EBCにおいては,被説明項で記述される現象がどのような振る舞いをするかが,説明項で記述される真理によって制約されるのである.例えばいちご配り事例では,母親の行動は〈23は3で割り切れない〉という(数学的)真理によって制約される.つまり,母親は23個のいちごを3人の子供に均等に配り切るという行動はとれないのである.そしてこの点に関連して,EBCの被説明項はある種の\emph{不可避性}を持っている.再びいちご配りの事例で,母親は単にいちごを均等に配れなかっただけでなく,配り切ることは\emph{できなかった},配りきれないことは不可避だったのである.

テーゼAは,EBCと因果的説明がかなり異なる仕方で働いていることをはっきりさせてくれるという点で重要である.というのも,被説明項を\emph{生じさせた原因}を特定する,という仕方で因果的説明は働くのに対して,被説明項を\emph{支配する制約}を特定する,という仕方でEBCは働く,ということになるからである\footnote{
この対比は,ボトムアップの説明とトップダウンの説明の対比とよく似ている\citep{kitcher1985two}.ほげほげ.
}.

\subsubsection{テーゼB}\label{b}
テーゼBは,制約の\emph{働き}を明確化するものだと言える.そして制約は,被説明項が比較的強い必然性でもって不可避であるのが何に由来するのかを特定するものだとされる.例えばいちご配り事例では,母親のいちご配りの試みがある程度の必然性でもって必ず失敗するということが,〈23は3で割り切れない〉という制約に由来することが特定されている.

しかし,必然性が「比較的強い」とはどういうことか? ここでは,必然性についてのLangeのやや独自のアイデアが用いられていることに注意しよう.Langeによれば,必然性にはいくらかの種類があり,それらは強さの観点から比較可能である.このアイデアは,次のような観察に基づいている.まず,偶然[accident]と法則の区別を考えよう.法則はある種の必然性を持つが,偶然は持たない.というのも,法則は反事実的に\myterm{頑強}{resilient}だが,偶然はそうではないからである.例えば,仮にここに(現実にはない)荷電粒子が存在したとしても,クーロンの法則は依然として成り立つだろう.つまり,クーロンの法則は反事実的に頑強である.一方で,仮にカップルが引っ越してきたとしたら,私が住むアパートの住人はみな単身者であるという偶然的事実は成り立たなくなるだろう.つまり,〈みな単身者である〉という偶然的事実は反事実的に頑強でない.「比較的強い必然性」というアイデアは,同じことをより高いレベルで行うことによる.例えば,仮に新たな,現実に成り立っていない架空の力の法則(クーロンの法則や運動法則とは異なる,架空の力の法則)が成り立っていたとしても,エネルギー保存則は成り立っていただろう.すなわち,エネルギー保存則という法則は,種々の力の法則よりも強い反事実的な頑強さを持っていることになる.換言すると,エネルギー保存則は,種々の力の法則と比較して,それらよりも強い必然性を持っているのである\citep[Ch. 1]{lange2009lawmakers}.これが「比較的強い必然性」というアイデアの内実である.

Langeはこの考えをEBCに応用する.第一に,制約としての説明項と,不可避性を記述する非説明項は,\emph{同じ}強さの必然性を持つ.特に,被説明項がある特定の強さの必然性を持っていることを,何らかの制約を説明項に持ってくることによって特定している\citep[25]{lange2016bwc}.例えばいちご配り事例では,〈23は3で割り切れない〉という制約が説明項であることによって,母親のいちご配りの失敗の必然性が数学的な強さの必然性であることが特定されている.そして第二に,被説明項が不可避であるのは,それが,比較的強い必然性を持った法則としての制約からある特別な仕方で\emph{導出される}ことのおかげであるとされる\citep[28--33]{Lange2018bwcsum}.ただし「特別な仕方」での導出とは,詳しくは\ref{objection}節で見るが,説明の非単調性を損なわないような仕方での導出のことである.

%制約が,強い法則のことだ,というのを補足.普通の因果法則が制約にはならない,ということをちゃんと説明する.

%エネルギー保存則→運動量保存則.

\subsubsection{テーゼC}\label{c}
最後にテーゼCは,必然性の強さが何によって決定されるのかを述べる.そしてLangeは,それは仮定法の事実のパターンによって決定される,と主張する.(「仮定法の事実」というのは,反事実的条件文------言語表現------によって表現されるような事実------世界で成り立っている非言語的な存在者------のことである.例えば「今日雨が降っていたら運動会は中止だっただろう」という反事実的条件文によって,今日雨が降っていたら運動会は中止だっただろう,という事実が表現される.)この見解はテーゼBよりもさらに独自性の強い主張である.というのも,このテーゼは,〈法則が反事実的条件法の真理を支える〉という標準的な考えを否定するものだからである.むしろLangeは,〈反事実的条件法の真理が法則を支える〉と言うのである.以下ではこの見解を明確化しよう.

Langeはこの見解を打ち出すにあたり,まず次の\myterm{sub-nomicな安定性}{sub-nomic stability}という性質を定義する\citep[29]{lange2009lawmakers}.
\begin{quote}
    $\Gamma$をsub-nomicな真理の集合とする.$\Gamma$は\emph{sub-nomicに安定}である\\
    $\Longleftrightarrow$ $\Gamma$と整合的な全てのsub-nomicな命題$p$について,全ての$m\in\Gamma$について,$p\necif m$.
\end{quote}
(ただし,命題$p$がsub-nomicであるとは,$p$が〈$\ldots$は法則である〉という形をして\emph{いない}ということである.例えば〈孤立系の力学的エネルギーの総量は変化しない〉というのはsub-nomicで,〈孤立系の力学的エネルギーの総量は変化しない,ということは法則である〉はsub-nomicではない.)重要なのは,Langeはこの定義を,単なる実質同値ではなく実在的な定義として意図している点である.すなわち,
\begin{quote}
    (テーゼC$^{-}$)\quad sub-nomicに安定な真理の集合よりも仮定法の事実のパターンの方が\emph{存在論的に先行する},
\end{quote}
ということである.
%わざわざsub-nomicだと言うのはどういうポイントがある?

そして,このように定義されるsub-nomicに安定な集合が,ある種類の強さを持つ必然的真理(すなわち法則)の集合と同一視されるのである.例えば,数学的真理の集合はsub-nomicに安定だろう.というのも,数学的真理の集合と整合的な全てのsub-nomicな命題$p$について,どの数学的真理$m$についても,$p\necif m$が成り立つだろうからである.例えば,$m$を「23は3で割り切れない」とすると,次のような仮定法の事実のパターンが想定できるだろう.
\begin{quote}
    私の住むアパートにカップルが引っ越してくる $\necif$ 23は3で割り切れない.

    クーロンの法則が成り立たない $\necif$ 23は3で割り切れない\footnote{
    正確には,〈二つの点電荷の間に働く静電気力の大きさは,それぞれの電気量の大きさの積に比例し,点電荷間の距離の二乗に反比例しない$\necif$23は3で割り切れない〉とすべきである.「クーロンの法則が成り立たない」は法則に明示的に言及しており,厳密に言うとsub-nomicではないからである.
    }.

    力学的エネルギー保存則が成り立たない $\necif$ 23は3で割り切れない\footnote{
    一つ前の注と同様の理由で,法則に明示的に言及しないように書き換えるべきである.
    }.

    $\quad\vdots$
\end{quote}
要するに,法則の必然性の強さは,それがどのsub-nomicに安定の集合に属するのかによって\emph{個別化される},ということである.別の言い方をすると,法則$m_1, m_2$が同じ強さの必然性を持つことは,それらが同じ一つのsub-nomicに安定な集合に属するということに他ならない.すると,このこととテーゼC$^{-}$を合わせると,
\begin{quote}
    (テーゼC)\quad 仮定法の事実のパターンが,法則がある種類の強さの必然性を持つことよりも\emph{存在論的に先立つ},
\end{quote}
ということになる.〈反事実的条件法の真理が法則を支える〉というのはこういうことである\footnote{
Langeはさらに,仮定法の事実が\myterm{存在論的なベッドロック}{ontological bedrock}である,つまり,仮定法の事実が存在論的に最も基礎的なカテゴリーであるという極めて独自の見解を打ち出している\citep[137]{lange2009lawmakers}.
}.

EBCの事例と特徴,そしてEBCに関してLangeの支持する三つのテーゼについては以上である.重要なことに,三つのテーゼを合わせると,\emph{EBCの説明力は,いかなる仮定法の事実が成り立つのかに由来する}ということが出てくる.次節と次々節ではこの点を問題視する.

\section{「いかにして」の問いと「何のおかげで」の問い}\label{howandwhat}
前節では,EBCの事例と,EBCについてのLangeの見解を見た.次節以降で,その見解に対する反論を私は展開するが,本節ではその準備として,方法論的な検討を簡単に行う.

初心に戻ると,EBCを巡る議論は,EBCという種類の説明の本性を巡る議論だと言える.そして,一般に種類$E$の説明についての学説を提示しようとする際,私たちは次の二つの問いを区別するべきである.
    \begin{description}
    \item[「いかにして」の問い:]〈$\phi$が$\psi$を$E$の意味で説明する〉について,\emph{いかにして}$\phi$は$\psi$を説明しているのか?
    \item[「何のおかげで」の問い:]〈$\phi$が$\psi$を$E$の意味で説明する〉について,\emph{何のおかげで}$\phi$は$\psi$を説明できているのか?
    \end{description}
「いかにして」の問いは,\emph{問題の種類の説明が示す際立った性質にどのようなものがあるのか}を問うている.例えば因果的説明という種類の説明(e.g., 「窓が割れた.なぜならば,石を投げたからだ.」)について,「いかにして」の問いに対しては「$\phi$が$\psi$を産出する[produce]という仕方で,$\phi$は$\psi$を説明している」や,「$\phi$が$\psi$に違いをもたらす[make a difference]という仕方で,$\phi$は$\psi$を説明している」というように答えられうる.一方で「何のおかげで」の問いは,\emph{問題の種類の説明の説明力は何に由来しているのか}を問うている.再び因果的説明を例に取ると,「何のおかげで」の問いに対しては「$\phi$が$\psi$の原因であるおかげで,$\phi$は$\psi$を説明している」と答えられうる.

「いかにして」の問いと「何のおかげで」の問いは混同されるべきではない.この二つの問いはそれぞれ,問題の種類の説明がどのような\emph{派生的性質}を持つのか,どのような\emph{本質的性質}を持つのか,を問うているのである.アナロジーとして,例えば水は,沸点が100$^\circ$Cである,4$^\circ$Cで密度が最大になる,といった派生的な性質を持っている.これらの性質は,水の分子構造が$\text{H}_2\text{O}$であるという本質的性質(特に水素結合という結合)に由来している.同様にして,任意の種類の説明は,なんらかの派生的性質と,それが由来するところの本質的性質を持つ.再び因果的説明という種類を取り上げると,その派生的性質(e.g., 産出性,差異形成)は,本質的性質(e.g., 因果情報を伝えること)に由来するのである\footnote{
もちろん,この因果的説明についての見解は単なる一例である.産出性は派生的性質ではないとか,あるいは産出性こそが本質的性質である,といった見解が正しい可能性は残されている.重要なのは,そのような可能性においても,「いかにして」の問いと「何のおかげで」の問いの区別は維持されている,ということである.
}.

直観的には,「何のおかげで」の問いは「いかにして」の問いよりも根本的であると言いたくなる.というのも,「いかにして」の問いへの答えは「何のおかげで」の問いへの答えに依存しているように見えるからだ.しかし少なくとも本論ではそこまでの主張を行う必要はない.重要なのは,問題の種類の説明についての学説を満足な形で提示するためには,\emph{どちらの}問いに対しても答えを与えなければならない,ということである.水の研究において,派生的性質と本質的性質のいずれの研究も重要であることは明らかだと思う.同様の態度を説明論にも向けるべきである.

さて,以上の方法論的な検討を踏まえて,Langeの見解に戻ろう.EBCという種類の説明について,Langeは「いかにして」の問いと「何のおかげで」の問いにどのようにして答えているだろうか? 私の考えでは,テーゼAが「いかにして」の問いへの答え,テーゼB+Cが「何のおかげで」の問いへの答えとして解釈されるべきである.順に確認しよう.まず,
\begin{quote}
    \emph{テーゼA}\quad EBCでは,説明項が制約を記述しており,それが被説明項で記述される出来事の不可避性を導いている,
\end{quote}
は,「いかにして」の問いへの答えだと解釈するのが穏当だろう.というのも,このテーゼは,EBCでは説明項が被説明項を制約するという\emph{仕方で}説明がなされるのだ,と述べているからである.

続いて,
\begin{quote}
    \emph{テーゼB}\quad 説明項で記述される制約は,被説明項で記述される出来事の不可避性------すなわち,比較的強い必然性------が何に由来するのかを特定している,

    \emph{テーゼC}\quad 必然性の強さは,いかなる仮定法の事実が成り立っているかによって決定される,
\end{quote}
は,これら二つのテーゼを合わせて,「何のおかげで」の問いへの答えとして理解されるべきだろう.
%というのも,制約というのは様相的な概念だからである.例えばチェスのルールが私の指し手を制約するというのは,そのルールが指しては\emph{ならない}手を指定しているということである.私は,ナイトを口に入れるという手を指すことは\emph{できない}のである.従ってテーゼBは,制約は\emph{いかにして}様相的だと言えるのかを,より明確に述べたものだと言えよう.すなわち,制約は,「普通の因果法則」よりも強い必然性を持っている,という仕方で様相的なのである.
というのも,説明項と被説明項の持つ必然性が同じであり,かつ説明項が被説明項を特別な仕方で導出することのおかげでEBCの意味での説明は働いており(テーゼB),そしてそれらの持つ必然性が同じであることは仮定法の事実のパターンによって決定される(テーゼC)からである.すなわち,EBCの説明力の源泉は,特別な仕方での導出関係と仮定法の事実のパターンとの二つに\emph{由来している}のだと,これら二つのテーゼは主張している.このことは「何のおかげで」の問いへの答えとして理解すべきだろう.

このように理解されるLangeの見解に対して,以下,私は二種類の反論を提示する.
%それは,Langeの「いかにして」の問いへの答えと,「何のおかげで」の問いへの答えが不整合だ,という反論である.換言すると,テーゼA+BとテーゼCが不整合なのである.これらが不整合であることは\ref{objection}節で示す.その節での議論がうまくいっているとすれば,Langeの見解は修正されなければならない.Langeには,テーゼA+Bを捨てるという選択肢とテーゼCを捨てるという選択肢があるが,本論ではテーゼCを捨てるという選択肢の可能性を追求する.すなわち,\ref{ebcbyground}節で私は,「何のおかげで」の問いへの答えを備えている,EBCについての理論を,\myterm{グラウンディング}{grounding}の概念を用いて構築する.

\section{Langeへの反論}\label{objection}
Langeへの反論を行おう.一つ目は,テーゼCに対する反論である.ただし,この反論はいくつかの条件つきで,決定的なものではない.二つ目は,テーゼBに対する反論である.\ref{ebcbyground}節でも述べるように,この反論の方が,Langeと私の見解のより根本的な対立を示している.

\subsection{テーゼCへの反論}\label{objectiontoc}
テーゼCへの反論は次のようなものである.もし (i) 「仮定法の事実が必然性の強さを決定する」と言うときの決定関係が一種の説明関係であり,かつ (ii) その説明と制約による説明が同じ意味で説明であり,かつ (iii) 事実の不可避性が反事実的条件文によって表現されるものであるのだとしたら,テーゼCは悪しき自己説明に陥る.順を追ってこの反論を明確化・擁護しよう.

条件 (i) について.Langeによれば,法則の必然性の強さはそれが属するsub-nomicに安定の集合によって決定され,その集合は基礎的な仮定法の事実のパターンによって決定されるのだった.この決定関係は,ある種の形而上学的な説明関係だと言えるかもしれない.このことは,「なぜこの法則はしかじかの強さの必然性を持つのか?」という問いに対して,Langeに従えば「基礎的な仮定法の事実のパターンがかくかくのようになっているからだ」と答えられるように見えることに示唆される.

条件 (ii) について.仮に決定関係が説明関係の一種だとすると,それはEBCと次の重要な意味で同じ説明だと言えるかもしれない.つまり,二つの説明を「繋ぐ」ことができる,という意味で同じかもしれない.もしそうだとすると,いま基礎的な仮定法の事実のパターンが制約の必然性の強さを説明し,そして制約の必然性の強さが被説明項の不可避性を説明しているため,これらを「繋げ」て,基礎的な仮定法の事実のパターンが,結局のところは被説明項の不可避性を説明する,と言えることになる.

条件 (iii) について.ある結果が不可避だということは\emph{反事実的条件文}によって表現される事柄であるように思われる.どういうことか.いちご配り事例を例に取ると,
\begin{exe}
    \ex 母親が23個のいちごを3人の子供に均等に配るのに失敗する,ということは不可避である,
\end{exe}
というのは,
\begin{exe}
    \ex 何が起きた\emph{としても},母親は23個のいちごを3人の子供に均等に配るのに失敗した\emph{だろう},
\end{exe}
と(大まかに言えば)同値だろう.$\phi$であることが不可避だというのは,他のどのような反事実的な状況においても$\phi$が生じてしまう,ということだからである.実際,Lange自身もこのことには同意すると思われる.というのも,Langeも,EBCの直観的な特徴づけとして次のように述べているからである.
\begin{quote}
    [制約が数学的真理であるようなEBC]は\footnote{
    私が補った「制約が数学的真理であるようなEBC」は,原文では``A distinctively mathematical explanation''と書かれている箇所である.
    }むしろ,おおよそ,説明される事実がいかにして\myterm{別様ではありえなかっただろうか}{chould not have been otherwise}------実際,因果的パワーの作用から結果しうるよりも強い程度に不可避であるということ------を示すことによって働くのである[$\ldots$].\citep[強調筆者,][5--6]{lange2016bwc}
\end{quote}
以上の記述の中の私が強調した箇所では,不可避性に反事実的なニュアンスがあることがたしかに認められている.

以上,まず (iii) より,EBCの被説明項は全称量化された仮定法の事実である.これは換言すると,EBCの被説明項は仮定法の事実のパターンだということである(このパターンを\emph{不可避性仮定法事実パターン}と呼ぼう).そして (i), (ii) より,この不可避性仮定法事実パターンが,結局のところは基礎的な仮定法の事実のパターン(\emph{基礎的仮定法事実パターン}と呼ぼう)によって説明されることになる.これは仮定法の事実のパターンを仮定法の事実のパターンによって説明することであり,自己説明である.

ただし,Langeにはまだ反論の余地が残されているように見えるかもしれない.というのも,二つの仮定法の事実のパターンが異なるパターンで,従って実は問題の説明は自己説明ではないという可能性があるからである.しかし実はその可能性は無い.なぜならば,テーゼBを仮定すると,不可避性仮定法事実パターンが基礎的仮定法事実パターンに包摂されていると分かるからである.この点を明確にするために,再びいちご配り事例を考えよう.まず,「母親は23個のいちごを3人の子供に均等に配るのに失敗した」という文を「$e$」で表すとすると,(2) の文は,次のように量化された反事実的条件文として書ける\footnote{
本論では命題への量化を許す.これに伴う問題は本論の文脈では無視できるはずである.
}.
\begin{align*}
    \forall p ( p\necif e) .
\end{align*}
これはつまり,次のような反事実的条件文で表されるような仮定法の事実のパターンを指示している.
\begin{quote}
    $明日,雨が降る\necif e.$

    $私のアパートにカップルが引っ越してくる\necif e.$

    $クーロンの法則が成り立たない\necif e.$

    $\vdots$
\end{quote}
ここで,「23は3で割り切れない」を$c$と書くことにする.テーゼBによれば,EBCの被説明項の$e$と説明項の$c$は同じ強さの必然性を持つ,すなわち,同じsub-nomicに安定の集合$S$に属する.そして$S$の外延は,\ref{c}節で確かめたsub-nomicな安定性の定義に従って,基礎的仮定法事実パターンによって決定される.そしてそのパターンは次のようになっているはずである.
\begin{quote}
    $明日,雨が降る\necif c.$

    $私のアパートにカップルが引っ越してくる\necif c.$

    $クーロンの法則が成り立たない\necif c.$

    $\vdots$
    
    $明日,雨が降る\necif e.$

    $私のアパートにカップルが引っ越してくる\necif e.$

    $クーロンの法則が成り立たない\necif e.$

    $\vdots$
\end{quote}
一見して明らかな通り,この基礎的仮定法事実パターンは,不可避性仮定法事実パターンを包摂している.従って,基礎的仮定法事実パターンによって不可避性仮定法事実パターンを説明することは,被説明項を被説明項自身(正確には,被説明項よりも巨大なパターン)によって説明することであり,悪しき自己説明である\footnote{
もしかするとLangeは,\emph{まさに}このタイプの説明を念頭に置き,それで良しとしているのかもしれない.つまり,あるパターン(ここでは不可避性仮定法事実パターン)を,より巨大なパターン(ここでは基礎的仮定法事実パターン)に包摂するということが説明することなのだ,と考えているのかもしれない.本論ではこの考えの妥当性を検討する余裕はない.だが,仮にこの考えが妥当だったとしても,まだ二つ目の反論(\ref{objectiontob}節)がLangeには向けられている.
}.

%問題は,簡潔に言うと,\emph{不可避性が反事実的条件文によって表現されるとして,さらにテーゼAを仮定すると,テーゼCによる「何のおかげで」の問いへの答えは悪しき循環に陥る}というものである.より詳しく言うと,問題は以下のように生じる.反事実的条件文は,仮定法の事実を表現する文である.従って,EBCの被説明項は結局のところ,特別な種類の仮定法の事実だと言える.正確には,「何が起きたとしても」という全称量化がかけられた仮定法の事実が,EBCの被説明項である.よって,
%\begin{quote}
%    (HOW)\quad EBCは,ある制約がその種の仮定法の事実が成り立つことを制約する,という仕方で働く説明である,
%\end{quote}
%と言うことができる.(HOW) は,不可避性が反事実的条件文によって表現されるということとテーゼAから含意される.そして,(HOW) はテーゼCと不整合である.というのも,制約の説明力はいかなる仮定法の事実のパターンが成り立っているかに由来する,というのがテーゼCだったからである.つまり,(HOW) に従うとある特別な種類の仮定法の事実が成り立つことが制約によって説明され,そしてテーゼCに従うと,その制約の説明力が今度は仮定法の事実のパターンに由来することになるのである.これは結局のところ,「何のおかげで,制約は仮定法の事実を説明できるのか?」という「何のおかげで」の問いに対して,仮定法の事実のパターンに訴えて答えようとすることであり,悪しき循環である.
%subjunctive < const < subjunctive ではなく,subjunctive < (const < subjunctive) なのでは?



\subsection{テーゼBへの反論}\label{objectiontob}
テーゼBへの反論は次のようなものである.

\subsection{二つの反論を踏まえて}
以上の議論により,私はテーゼA+BとテーゼCは不整合であると結論づける.今,私たちが取れる選択肢としては,(1) テーゼA+Bを捨てるという選択肢と (2) テーゼCを捨てるという選択肢がある.どちらが望ましいのかについて,本論で決定的な議論を行うことはできない.しかし,私は選択肢 (2) が魅力的だと思う.というのも,LangeによるEBCに関する「いかにして」の問いへの答え(テーゼA+B)は非常にカラフルでもっともらしい------特に,制約,不可避性,法則等の諸概念の緊密なつながりを見事に描き出しているように思われる------のに対して,「何のおかげで」の問いへの答え(テーゼC)は論争的な形而上学的仮定をいくつか置いているように思われるからである.とりわけ,法則よりも反事実的条件文の真理の方が基礎的だという形而上学的仮定は論争的である\citep{Demarest2012counter,woodwardetal2011lange}.もちろん,論争的な仮定が必ず誤りだというわけではないが,本論ではより安全な路線を追求してみようということである.

\section{グラウンディングによるEBCの理論}\label{ebcbyground}
私は,「何のおかげで」の問いへの答えを備えたEBCの理論を提案する.それは\emph{グラウンディング}という形而上学的な依存関係に基づいた理論である.

\subsection{グラウンディングの概念}
まずはグラウンディングの概念を導入しよう.グラウンディングとは,非因果的な依存関係のことである\citep[see][]{fine2001realism,schaffer2009grounds,rosen2010dependence}.次の例文は,グラウンディング関係のインスタンスを表している.
\begin{exe}
    \ex
        \begin{xlist}
            \ex 薫が痛みを感じているのは,薫の脳がしかじかの状態にあるおかげである.\label{pain}
            \ex この紙切れは千円札である.なぜならば,それは国立印刷局で印刷されているからだ.
            \ex $A\land B$が真であるのは,$A$と$B$が両方真であることによる.
        \end{xlist}
\end{exe}
例えば (\ref{pain}) は,〈薫の脳状態が薫の痛み状態をグラウンドする〉というグラウンディング関係のインスタンスを表している.この関係は\emph{依存関係}である.つまり,痛み状態は脳状態に依存している.このことは,もし薫の脳状態が違っていれば薫の痛み状態も変わっていただろう,ということから示唆される.しかし,この関係は\emph{因果関係}ではない.つまり,脳状態は痛み状態を引き起こしているわけではない.このことは,脳状態と痛み状態との間に時間的なギャップが無いことなどから示唆される.むしろ,脳状態は痛み状態を\myterm{決定}{determine}ないし\myterm{構成}{constitute}していると言えるだろう.この決定関係(ないし構成関係)がグラウンディング関係である.

さらに,\myterm{完全な}{full}グラウンディングと\myterm{部分的な}{partial}グラウンディングとが区別される\citep{fine2012guide}.事実の集合$\Gamma$が事実$\phi$を完全にグラウンドするとは,$\Gamma$が$\phi$を決定するのに十分だということである.対して,事実$\psi$が事実$\phi$を部分的にグラウンドするとは,$\psi$が$\phi$を決定するのに貢献するということである.例えば,〈イアン・マッケイが歌っている〉は〈FUGAZIが演奏中である〉を部分的にグラウンドするだろう.さらに,〈ジョー・ラリーがベースを弾いている〉等の事実を適切に付け加えることで,完全なグラウンディング言明を構成できる.

完全なグラウンディングは文オペレータ$<$,部分的なグラウンディングは$\prec$として定式化される\cite{fine2012guide}.$<$は多対一のオペレータで,$\prec$は一対一のオペレータである.先の例で言えば,
\begin{itemize}
    \item イアン・マッケイが歌っている $\prec$ FUGAZIが演奏中である,
    \item イアン・マッケイが歌っている, ジョー・ラリーがベースを弾いている, $\ldots<$ FUGAZIが演奏中である,
\end{itemize}
というように書ける.

さらに,グラウンディングに関する標準的な原理を確認しておこう.私が以下で利用するのは,次の原理である.
\begin{quote}
    (CUT)\quad $\Gamma<\phi, \phi<\psi \vDash \Gamma<\psi$

    ($\forall$-Grounding)\quad $\phi(a), \phi(b), \ldots < \forall x \phi(x)$

    (CF)\quad $(\phi<\psi)\supset(\phi\necif\psi)$
\end{quote}
さらに,反事実的条件法に関する原理として,
\begin{quote}
    (IRRELEVANCE)\quad $\phi\necif\psi, \chi と \{\phi,\psi\} は因果的に無関係 < \phi\land\chi\necif\psi$
\end{quote}

\subsection{グラウンドによる説明としての制約による説明}
前小節で導入されたグラウンディングの概念を用いて,EBCの理論を構築しよう.結論から述べると,「何のおかげで」の問いへの答えは次の通りである.(「\underline{I}n \underline{V}irtue \underline{O}f \underline{W}hat」で「IVOW」である.)
\begin{quote}
    (IVOW) $\phi$が$\psi$をEBCの意味で説明するのは,$\phi\prec\psi$であるおかげである.
\end{quote}
(完全なグラウンディング$<$ではなく部分的なグラウンディング$\prec$であることに注意せよ.)すなわち,EBCの説明力は,世界で生じているグラウンディング関係を特定することに由来しているのである.これは,因果的説明の説明力が,世界で生じている因果関係を特定することに由来するのと対称的である.この対称性は,私のEBCの理論の理論的美徳である.

\subsubsection{形式化}
さて,以下では (IVOW) を支持する議論を行う.議論は,いちご配り事例を例にとって,実際に部分的なグラウンディング関係を構成することによる.そこで,(IVOW) をいちご配り事例に適用するために,いちご配り事例を少し形式化しよう.まず,
\begin{quote}
\begin{description}
    \item[$p:$] 母親は子供を3人,いちごを23個持つ.
    \item[$e:$] 母親はいちごを子供に均等に配るのに失敗する.
    \item[$c:$] 23は3で割り切れない.
\end{description}
\end{quote}
とする.そして,「$\ldots$であることは制約である」を「$Const(\ldots)$」と略記する.さらに,\ref{objection}節で指摘した通り,EBCの被説明項は,次のような全称量化がかかった反事実的条件文である.
\begin{exe}
    \ex 何が起きたとしても,母親は23個のいちごを3人の子供に均等に配るのに失敗しただろう.
\end{exe}
これは次のように形式化できるかもしれない.
\begin{gather*}
    \forall q ( p \land q \necif e )
\end{gather*}
ただし,より正確には,量化のドメインを制限しなければならない.ここでドメインは,前提$p$,結果$e$,制約$c$に「影響しない」命題の集合に制限されるべきだろう.というのも,「説明の舞台設定を崩さないような」どんなことが起こっても$e$が生じる,というのが,$e$が不可避だということだからである\footnote{
\citet[131--2]{skow2016reasons}も同様の趣旨のことを述べている.「``cannots''は不可能性をしるしづけるのであり,そして表面的には,「[母親の]いちごが彼女の子供たちに均等に配られなかったのは,そうすることができなかったからだ」という文はいちごが均等に配られるのが無条件に不可能だったと言っているように見える.しかしもちろんそうではないし,誰もそうだとは言わないだろう.いちごを均等に配ることは「条件付きで」不可能であるにすぎない.つまり,彼女が3人の子供と23個のいちごを持っている,という事実を固定したとしたら,不可能なのである.」
}.そこで,集合$Q$を次のように定義しよう.
\begin{align*}
    Q&=\{x: xはp,e,cに影響しない命題である\}\\
    &=\{x: xは前提p,結果e,制約cと論理的に整合的な命題である\}
\end{align*}
この$Q$をドメインの制限として,いちご配り事例の被説明項は次のように形式化される.
\begin{gather*}
    (\forall q\in Q) ( p \land q \necif e )
\end{gather*}
以上から,いちご配り事例の説明全体は,次のように形式化される.
\begin{gather*}
    Const(c)が(\forall q\in Q) ( p \land q \necif e )をEBCの意味で説明する.
\end{gather*}
この形式化に基づいて (IVOW) をいちご配り事例に適用すると次のようになる.
\begin{quote}
    (IVOW-S) $Const(c)が(\forall q\in Q) ( p \land q \necif e )をEBCの意味で説明するのは,Const(c) \prec (\forall q\in Q) ( p \land q \necif e )であるおかげである.$
\end{quote}

\subsubsection{論証}
従って,私は$Const(c) \prec (\forall q\in Q) ( p \land q \necif e )$という部分的なグラウンディング関係が成り立つことを示せばよい.これを示すために私が置く実質的な仮定は次の二つのみである.
\begin{quote}
    (Anti-Langean) $l が法則である < lに関連する反事実的条件文 \phi\necif\psi が成り立つ.$

    (Set) 任意の集合$S$とそのメンバー$a$について,$Sが存在する<a\in S$
\end{quote}
(Anti-Langean) は,〈反事実的条件文の真理が法則を支える〉とするLangeのテーゼCをひっくり返したものである.この仮定は,本論文においては比較的容易に正当化されよう.というのも,まさに\ref{objection}節で,(テーゼA+Bを認めた上では)テーゼCが成り立たないことを確かめたからである.よって,〈法則が反事実的条件文の真理を支える〉または〈法則と反事実的条件文の間には,どちらがどちらを支えるという関係は成り立たない〉のいずれかを認めなければならない.そして,法則と反事実的条件文の間に重要なつながりがあることを認めないという後者はもっともらしくない.従って前者の〈法則が反事実的条件文の真理を支える〉,すなわち (Anti-Langean) を私たちは受け入れるべきである.

(Set) は,集合のメンバーシップ関係は,その集合の本質だけによって決定される,と述べている.これは実質的な主張だが,やはり比較的容易に正当化されよう.(Set) が奇妙に見えるとしたら,それはメンバー$a$が,グラウンディングオペレータ$<$の右辺に突然現れているように見えることによるだろう.しかし,メンバー$a$についての情報は,集合$S$の本質に「すでに含まれている」と言うことができる.というのも,集合の同一性は次の原理に従うからである.
\begin{quote}
    (Extensionality) $S=T \Longleftrightarrow \forall x \forall y ( x\in S\equiv y\in T)$
\end{quote}
すると,集合$S=\{a, b, \ldots\}$の存在はそのメンバーの存在にグラウンドされる,つまり次の原理が成り立つと考えるのが自然だろう.
\begin{quote}
    $aが存在する, bが存在する, \ldots < Sが存在する$
\end{quote}
これはすなわち,$S$の存在の中に,すでに$a$の存在についての情報が含まれていることを意味する.以上のようにして (Set) の奇妙さは解消される.

さて,(Anti-Langean) と (Set) を仮定すると,$Const(c) \prec (\forall q\in Q) ( p \land q \necif e )$は次のようにして導出される.

\begin{prop}\label{prop}
$Const(c) \prec (\forall q\in Q) ( p \land q \necif e )$ である.
\begin{proof}
まず,(Anti-Langean) により,
\begin{gather}
    Const(c)<(p\necif e) \label{antilangean}
\end{gather}
である.これは,制約が特別な種類の法則であることによる.また,〈23は3で割り切れない〉という法則が,〈母親が子供3人,いちご23個を持っていたら,いちご配りの試みは失敗していただろう〉という反事実的条件文の真理を支える,というのは (Anti-Langean) を認めるならば自明だろう.次に,各$q_i\in Q$について,(IRRELEVANCE) により,
\begin{gather}
    p\necif e, q_i\in Q < p\land q_i\necif e \label{irrelevance}
\end{gather}
である.というのも,$q_i\in Q$であることは,定義より,$q_i$が$p, e$と因果的に無関係であることにほかならないからである.%従って,
($\forall$-Grounding) より,
\begin{gather}
    p\land q_1\necif e, p\land q_2\necif e, \ldots < (\forall q\in Q)(p\land q\necif e) \label{forall}
\end{gather}
となる.すると,(\ref{antilangean},\ref{irrelevance},\ref{forall}) と (CUT) により,
\begin{gather}
    Const(c), q_1\in Q, q_2\in Q, \ldots < (\forall q\in Q)(p\land q\necif e) \label{cut1}
\end{gather}
を得る.さらに (SET) と (CUT) によって,
\begin{gather}
    Const(c), Qが存在する < (\forall q\in Q)(p\land q\necif e) \label{set}
\end{gather}
である.ここで,$Q$の存在は,定義より,命題$p, e, c$にグラウンドされる(つまり, 「$p, e, c < Q が存在する$」が真)と言っていいだろう.従って,これと (\ref{set}) と (CUT) より,
\begin{gather}
    Const(c), p, e, c < (\forall q\in Q)(p\land q\necif e) \label{pec}
\end{gather}
となる.すると,(\ref{pec}) と完全なグラウディング/部分的なグラウンディングの定義により,所望の,
\begin{gather}
    Const(c) \prec (\forall q\in Q)(p\land q\necif e) \label{desired}
\end{gather}
が得られる.
\end{proof}
\end{prop}

\begin{align*}
    &(1)& &Const(c)<(p\necif e) \tag{Anti-Langeanより} \\
    &(2)& &p\necif e, q_i\in Q < p\land q_i\necif e \quad \quad (各q_i\in Qについて)\tag{Irrelevanceより} \\
    &(3)& &p\land q_1\necif e, p\land q_2\necif e, \ldots < (\forall q\in Q)(p\land q\necif e) \tag{$\forall$-Groundingより} \\
    &(4)& &Const(c), q_1\in Q, q_2\in Q, \ldots < (\forall q\in Q)(p\land q\necif e) \tag{1, 2, 3, Cutより} \\
    &(5)& &Const(c), Qが存在する < (\forall q\in Q)(p\land q\necif e) \tag{Set, Cutにより} \\
    &(6)& &Const(c), p, e, c < (\forall q\in Q)(p\land q\necif e) \tag{Cutより} \\
    &(7)& &Const(c) \prec (\forall q\in Q)(p\land q\necif e) \tag{6, Subsumptionより}
\end{align*}

以上の(準形式的な)証明は,次のようなダイアグラムとして描ける.ただし,二重矢印が完全なグラウディング,一重矢印が部分的なグラウンディングを表す.

\scriptsize
\begin{center}
\begin{tikzpicture}%[every node/.style={draw,circle}]
\node (a) at (0,0) {$p, e, c$};
\node (b) at (3,0) {$Qが存在する$};
\node (c) at (6,0) {$q_1\in Q$};
\node (d) at (3,2) {$Const(c)$};
\node (e) at (6,2) {$p\necif e$};
\node (f) at (9,1) {$p\land q_1\necif e$};
\node (g) at (12,1) {$p\land q_2\necif e$};
\node (h) at (11.5,2.5) {$\cdots$};
\node (i) at (13.5,1) {$\cdots$};
\node (j) at (9,4) {$(\forall q\in Q)(p\land q\necif e)$};
\node (k) at (11,-0.5) {$\ldots$};
\node (l) at (13,-0.5) {$\ldots$};

\draw[->,double distance=2pt] (a) -- (b);
\draw[->,double distance=2pt] (b) -- (c);
\draw[->,double distance=2pt] (d) -- (e);
\draw[->] (c) -- (f);
\draw[->] (e) -- (f);
\draw[->] (f) -- (j);
\draw[->] (g) -- (j);
\draw[->] (k) -- (g);
\draw[->] (l) -- (g);
\end{tikzpicture}
\end{center}
\normalsize


\subsubsection{哲学的反省}
上の準形式的な証明とその結果は,哲学的に見てどのような興味深い点があるだろうか? 三つの点を指摘して,本節を終えよう.

一点目.命題\ref{prop}において,$Const(c)$は$(\forall q\in Q)(p\land q\necif e)$の完全なグラウンドではなく,部分的なグラウンドである.つまり,〈23が3で割り切れない〉というのが制約であることは,〈母親が23個のいちごを3人の子供に均等に配るのに失敗する〉というのが不可避だ,ということの部分的なグラウンドである.これを完全なグラウンディングの言明にするには,(\ref{pec}) より,さらに$p, e, c$------つまり,〈母親は23個のいちごと3人の子供を持つ〉という前提と,〈母親はいちご配りに実際失敗した〉という結果と,〈23は3で割り切れない〉という数学的真理------をそのグラウンドに追加しなければならない.このことは,換言すると,「$Const(c)$が$(\forall q\in Q)(p\land q\necif e)$をEBCの意味で説明する」という説明は,\emph{実は完全な説明ではない},つまり必要な情報を全て備えた説明にはなっていない,ということである.完全な説明にするには,$p, e, c$の情報も足さなければならない.

この帰結が出てきてしまうことは,私の見解を反駁する根拠になるように見えるかもしれない.しかし,むしろこの帰結は好ましい帰結だと言える.というのも,この帰結は,EBCの\emph{高階性}を反映していると解釈できるからである.ほげほげ.

二点目.グラウンドと反事実的条件法と法則のつながりは重要.

三点目.制約が特別な種類の法則だ,というLangeのアイデアを引き継げるか? プランは三通り.(1) 必然性の階層構造という考えをまるっきり捨てる.つまりLangeのアイデアを引き継がない.(2) 必然性の階層構造というアイデアを引き受ける.代わりに,法則がそのように強さの点で順序付けられるということをプリミティブとする.(Langeのように仮定法の事実には訴えられない.)(3) ローカルな階層構造を認める,という中道をゆく.$p, e,c$という説明の舞台設定が揃うと,どの法則が最も強いかが決まる,とする.つまり,法則の強さは文脈依存的とする.

\section{おわりに}
本論では,制約による説明がについてのLangeの議論を批判的に検討した.その結果,Langeの議論は,「いかにして」の問いへの答えとしては優れたものだが,「何のおかげで」の問いへの答えとしては満足の行くものではないことが示された(\ref{objection}節).そこで私は,「何のおかげで」の問いへの答えを備えた,制約による説明についての新しい見解------グラウンディングの概念を用いたもの------を提示した(\ref{ebcbyground}節).この見解は,制約による説明の重要な諸特徴をよく捉えている上に,Langeの見解の瑕疵を克服しているものであった.さらに,グラウンディング研究へのフィードバックもいくつか得られた.
%「speciality問題」に関しては,Langeに分があるかもしれない\citep[Ch.4]{lange2009lawmakers}.グラウンディング路線は,なぜ法則がspecialなのかには説明は与えられないかも.そこは支払うべき代償とする.

\begin{comment}

\section{メモ}
\cite{Lange2009a,Lange2009b,Lange2010,Lange2011,Lange2012,Lange2013a,Lange2013b,Lange2013c,Lange2014,Lange2015,Lange2016,Lange2018a,Lange2018b}

\subsection{Laws and Lawmakers}
偶然[accident]と法則[law]の違い\citep[13]{Lange2009lawmakers}:
\begin{itemize}
    \item 法則を破る反事実的条件法の前件は,何らかの法則と不整合.
    \item 偶然を破る反事実的条件法の前件は,どの法則とも整合的でありうる.
\end{itemize}

Langeの法則観の最もコアにある原理,Nomic Preservationも出てくる\citep[13]{Lange2009lawmakers}:
\begin{quote}
    NP\quad $m$は法則である iff\\
    全ての法則と論理的に整合的であるような任意の反事実的な前提$p$のもとで,$m$は依然として成り立つだろう.
\end{quote}
ただしNPは修正が必要.
\begin{enumerate}
    \item 関連性問題(14).
    \item 文脈問題(14--5).
    \item 法則と論理的真理の区別問題(15--6).
    \item taken together問題(16--7).
    \item sub-nomic問題(17--20).
\end{enumerate}
これら全部に対処すると,こうなる\citep[20]{Lange2009lawmakers}:
\begin{quote}
    NP\quad $m$は法則である iff \\
    任意の文脈で,任意の$p$について,法則の集合$N$が存在して,\\
    $p$は全ての$n\in N$とあわせて論理的に整合的であり,かつ,$p\Box\rightarrow m$.
\end{quote}

さらにもう少し修正が必要.
\begin{itemize}
    \item 
\end{itemize}

Langeは,NPが法則の\emph{外延}を捉えてはいるかもしれないが,\emph{説明}はできてないのではないか,という点にちゃんと気づいている\citep[Sect.1,6]{Lange2009lawmakers}

\emph{sub-nomic stability}の定義\citep[29]{Lange2009lawmakers}:
\begin{dfn}
    $\Gamma$: sub-nomicな集合.\\
    $\Gamma$はsub-nomically stable\\
    $\Longleftrightarrow$ すべての文脈で,\\
    $\Gamma$と整合する全てのsub-nomicな$p$について,\\
    すべての$m\in\Gamma$について,\\
    $p\Box\rightarrow m$
\end{dfn}
こう定義した上で,\emph{法則全体からなる集合}$\Lambda$を,\emph{極大でない,最大のsub-nomicな集合}と定義する.こうすると,循環性の問題は回避される.

プライオリティ問題への言及もあり!!
\begin{quote}
    \emph{プライオリティ問題:}法則より反事実のほうが基礎的,というのはおかしいのでは?
\end{quote}
\citet[31]{Lange2009lawmakers}は,そもそもどっちが基礎的かという話はしていない.
\begin{quote}
    この章では,法則と反事実の間の特別な関係を\emph{同定}することにかかわっているだけで,この関係が\emph{なぜ}成り立つかを理解することにかかわっているわけではない.つまり,法則が反事実に責任を負っているか,反事実が法則に責任を負っているか,あるいはそれらのどちらでもないか,ということにはかかわっていない.こうした問題には2章と4章で取り組む.(31)
\end{quote}

\begin{quote}
    自然法則とは何であるかについてのいかなる形而上学的学説も,法則が複数の層をなす可能性の余地を残すべきである.\citep[41]{Lange2009lawmakers}
\end{quote}

\subsection{\cite{Lange2011}. Conservation Laws in Scientific Explanations}
BWTのCh.2の前半部分の元ネタ(たぶん).

単なる偶然[coincidence]と制約[constraint]の違いの話をしている論文.特に,エネルギー保存則が,\emph{もし制約だったとしたら}それを使った説明の説明力の源泉は何か,を特定しようとしている.(それが実際に制約\emph{である},ということを示そうという論文ではない.)

%一般に,ある法則的言明には,\emph{ボトムアップの}説明と\emph{トップダウンの}説明が与えられうる.E.g., アルキメデスの原理には,力学的メカニズムに訴えるボトムアップの説明と,エネルギー保存則に訴えるトップダウンの説明が与えられる.

制約と偶然は,それぞれ次のように反事実によって定義される.
\begin{dfn}
エネルギー保存則が制約である\\
$\Longleftrightarrow$
現実にはない追加の力が存在する$\Box\hspace{-4pt}\rightarrow$エネルギーは保存される.\\
($A\Box\hspace{-4pt}\rightarrow C$)
\end{dfn}
\begin{dfn}
エネルギー保存則が偶然である\\
$\Longleftrightarrow$
現実にはない追加の力が存在する$\Diamond\hspace{-4pt}\rightarrow$エネルギーは保存されない.\\
($A\Diamond\hspace{-4pt}\rightarrow\lnot C\equiv \lnot(A\Box\hspace{-4pt}\rightarrow C)$)
\end{dfn}
制約の定義1で,反事実的条件法の前件に法則が入ってくることから,制約は\emph{メタ法則}だと言える\citep[170]{Lange2012}.

\subsection{\cite{Lange2012}. There Sweep}
BWTのCh.2の後半部分の元ネタ.

エネルギー保存則が制約なのか偶然なのかという区別は,何が何よりも説明的に優先するか[explanatorily prior]の区別\citep[157]{Lange2012}.
\begin{itemize}
	\item 制約であるなら,エネルギー保存則$<$個々の領域でのエネルギー保存.
	\item 偶然であるなら,個々の領域でのエネルギー保存$<$エネルギー保存則.
\end{itemize}

\emph{法則}の定義(暫定版)\citep[171--2]{Lange2012}:
\begin{dfn}
mが法則\\
$\Longleftrightarrow$ 全ての法則と論理的に整合的などんな命題pについても,($p\Box\hspace{-4pt}\rightarrow m$).
\end{dfn}


\emph{sub-nomic}と\emph{nomic}の区別\citep[171]{Lange2012}:
\begin{dfn}
pがsub-nomic\\
$\Longleftrightarrow$ pの真理メーカーが法則でない.
\end{dfn}
\begin{dfn}
pがnomic\\
$\Longleftrightarrow$ pの真理メーカーが法則.
\end{dfn}

\emph{sub-nomicな安定性}の定義\citep[172]{Lange2012}:
\begin{dfn}
$\Gamma$: sub-nomicな命題の集合で,$\Gamma=Cn\Gamma$.\\
$\Gamma$がsub-nomicな安定性を持つ\\
$\Longleftrightarrow(\forall m\in\Gamma)(\forall p)(\Gamma\cup\{p\}$は整合的$\rightarrow(\lnot(p\Diamond\hspace{-4pt}\rightarrow\lnot m))$.
\end{dfn}

\emph{法則}の定義(修正版)\citep[173]{Lange2012}:
\begin{dfn}
mが法則\\
$\Longleftrightarrow$ (極大でない)sub-nomicallyに安定した集合にmが属する.
\end{dfn}

\begin{thm}
sub-nomicallyに安定した集合のクラスは階層的である.\citep[173]{Lange2012}
\end{thm}

Langeの見解の利点\citep[174]{Lange2012}:
\begin{itemize}
	\item 様々なレベルの制約を認められる.
	\item 高階の法則としての制約の役割をうまく説明できる.(上の層のやつが下の層のやつを制約する.)
	\item 制約が種々の可能性を特徴づけるメカニズムの説明にもなってる.
\end{itemize}

\emph{nomicな安定性}の定義\citep[175]{Lange2012}:
\begin{dfn}
$\Gamma$: nomicないしsub-nomicな真理の集合で,$\Gamma=Cn\Gamma$.\\
$\Gamma$がnomicな安定性を持つ\\
$\Longleftrightarrow(\forall m\in\Gamma)(\forall p)(\Gamma\cup\{p\}は整合的\rightarrow (\lnot(p\Diamond\hspace{-4pt}\rightarrow\lnot m))$
\end{dfn}

最後の節で,本質主義的な法則観を退けている.(それは制約/偶然の区別をつけられないので.)

\subsection{\cite{Lange2013a}. What Makes a Scientific Explanation Distinctively Mathematical?}
BWCのCh.1の元ネタ.

非因果的説明を(ざっくり)定義している\citep[487,491]{Lange2013a}.
\begin{dfn}
    説明タイプEが非因果的である\\
    $\Longleftrightarrow$Eは,被説明項が生じるのがいかにして\emph{不可避}である(しかも,因果的必然性よりも強い意味で)のかを示すことによって説明している.
\end{dfn}
因果的説明も定義してる\citep[493]{Lange2013a}.
\begin{dfn}
    説明タイプEが因果的である\\
    $\Longleftrightarrow$ Eは,世界の因果情報を与えることによって説明している.
\end{dfn}

DMEが何\emph{ではない}か:
\begin{quote}
非因果的説明が非因果的だとされるのは,それが,その説明力を,世界の因果構造を記述していることから引き出しているわけではないからである\citep[495]{Lange2013a}.

DMEが被説明項の原因をたまたま特定していることがあることは,それが被説明項の原因を記述すること\emph{のおかげで}働いていることを意味しない.母親が三人の子供を持っていることがいちご配りの失敗を説明するのは,それが彼女の失敗の原因である\emph{おかげ}ではなく,彼女の成功を数学的に不可能にしている\emph{おかげ}である\citep[496]{Lange2013a}.

[DME]が原因と被説明項との間にinvokeするいかなるつながりも,普通の偶然的な自然法則の\emph{おかげで}成り立っているわけではなく,特に数学的必然性の\emph{おかげで}成り立っているのである.\citep[496--7]{Lange2013a}
\end{quote}

DMEが何\emph{であるか},あるいは,いかにして働いているか:
\begin{quote}
    科学におけるDMEに現れる自然法則は,存在する特定の種類の原因を記述するような法則を\emph{超越}していなければならない,と私は提案する.\citep[505]{Lange2013a}

    説明項となる事実が説明をするのにふさわしいものとなっているのは,それが通常の因果法則よりも様相的により必然的である(ニュートンの第二法則や数学的事実のように)こと\emph{のおかげ}である.あるいは,問題となっている物理的なタスクや配置を構成するwhyの問いのコンテクストにおいて理解されているおかげである.
    \begin{itemize}
        \item これは一つ目が法則が説明項に来るようなEBC(つまり,タイプcのEBC)? そして二つ目が,因果的な条件が説明項に来るようなEBC(つまりnタイプのEBC)?
    \end{itemize}
\end{quote}

\subsection{\cite{Lange2018b}.Because Without Cause}
Langeの「制約による説明」についての概説的な論文.

ほげーー
\end{comment}

\bibliographystyle{apa}
\bibliography{bibliography}

\end{document}  