\documentclass[twoside,14Q,dvipdfmx,uplatex]{jsarticle}
\pagestyle{myheadings}
\markboth{坪井------何が制約による説明を説明にしているのか?}{応用哲学会第15年次研究大会}
\setcounter{page}{1}
\title{何が制約による説明を説明にしているのか
\large{\\--- グラウンディングの観点から ---}
}
\author{坪井祥吾\thanks{一橋大学大学院社会学研究科,\texttt{tsuboishogo98@gmail.com}}}

\date{2023年4月22日}

%数式など
\usepackage{amsmath}
\usepackage{amsthm}
\usepackage{amssymb}
\usepackage{mathrsfs}

\usepackage{graphicx}

%url,ハイパーリンク
\usepackage{url}
\usepackage[dvipdfmx]{hyperref}

%引用スタイル
\usepackage{natbib}
\bibpunct[: ]{(}{)}{,}{a}{}{,}

%定理環境
\theoremstyle{definition}
\newtheorem{dfn}{定義}
\newtheorem{thm}{定理}
\newtheorem{lem}{補題}
\newtheorem{fact}{事実}
\newtheorem*{lem*}{}
\newtheorem{thesis}{テーゼ}

\usepackage{mathrsfs}
\usepackage{bussproofs} %証明図
\usepackage{fancybox}
\usepackage{tikz}
\usetikzlibrary{graphs}
\usetikzlibrary{positioning}
\allowdisplaybreaks[1]
\usepackage{url}

%圏点
\usepackage{pxrubrica}

%枠囲み
\usepackage{ascmac}
\usepackage{fancybox}

%tikz
\usepackage{tikz}
\usetikzlibrary{intersections, calc, arrows.meta}
\allowdisplaybreaks[1]

\begin{document}
\maketitle

\section{はじめに:科学的説明論とは何か}
なぜ恐竜は絶滅したのか?——巨大な小惑星がメキシコに衝突したからだ.なぜ2008年の大規模な不景気が生じたのか?——リーマン・ブラザーズが経営破綻したからだ.こうした説明を与えることは,科学という活動の重要な構成要素である.それゆえ,この種の説明は科学的説明[scientific explanation]と呼ばれ,科学哲学者の関心を集めてきた.そして科学哲学者は〈科学的説明とは何か?〉と問う.20世紀の終わり頃まで,この問いに対しては〈科学的説明とは因果的説明[causal explanation]のことである〉と答えるのが主流だった.つまり,ある出来事に科学的説明を与えるとは,その出来事の原因,ないしその出来事を成り立たせている因果メカニズムを特定することである,と.

しかし実際には,非因果的な種類の科学的説明が存在する(あるいは,少なくとも非因果的に見える説明が存在する)のである.この非因果的説明[non-causal explanation]は,近年の科学的説明論の一大トピックとなっている.たとえば,次のような説明が非因果的説明の典型例である.〈ツルツルした水平の床の上に,赤ちゃんの乗ったベビーカーが置いてある.外から力が加わらないとして,赤ちゃんがいくらベビーカーの中で動き回っても,ベビーカーはほとんど動かなかった.なぜか?——運動量保存則が成り立つからだ.〉この説明では運動量保存則が説明項になっているが,しかし一般に法則は出来事の原因にはなりえない.それゆえ,この説明は非因果的だと言われるのである.

以上から,科学的説明には二つのサブカテゴリーがあることが分かる.因果的説明と非因果的説明である.そして,科学的説明\emph{論}とは,こうした種々の説明についての理論を立てることを目指す分野だと言える.別の言い方をすると,説明に対する説明,つまり\emph{メタ説明}を与えるのを目指すのが,科学的説明論である.詳細は\ref{howandwhat}節で見るが,メタ説明としては次の二種類を区別するべきだ,というのが本論のコアのアイデアである.つまり,「いかにして」の説明と,「何のおかげで」の説明の区別である.

さて,本論の狙いは,非因果的説明のさらなるサブカテゴリー,\emph{制約による説明}[explanation by constraint]についての\cite{Lange2011,Lange2013a,Lange2016,Lange2018b}の理論を批判的に検討することである.詳細は\ref{explanationbyconstraint}節で見るが,制約による説明とは,被説明項が必然性を備えており,その必然性が,説明項で述べられる制約から導かれるような説明のことである.先程のベビーカーの例も,Langeによれば制約による説明である.というのも,ベビーカーはたまたま動かなかったのではなく動きえなかったのであり,その必然性は,運動量保存則という制約に由来しているからである.Langeはこの制約による説明についてのメタ説明を与えているが,それに対する私の批判は,(1) そのメタ説明は「いかにして」の説明にはなっているが「何のおかげで」の説明にはなっていないし,(2) 説明の文脈依存性も汲み取れていない,という二点である.特に (1) について,「何のおかげで」の説明を行うためには\emph{グラウンディング}[grounding]の概念を利用するのが望ましいと私は考えている.

以上を踏まえて,私が本論で擁護する理論を先に示してしまおう.
\begin{quote}
「$\phi_1, \phi_2, \ldots$は$\psi$を説明する」が非因果的説明であるのは,次の二つが成り立つ\emph{おかげ}である.
	\begin{enumerate}
	\item $\phi_1, \phi_2, \ldots$は$\psi$に関するグラウンディング情報$G$を含んでいる.	
	\item 「$\phi_1, \phi_2, \ldots$は$\psi$を説明する」の発話の文脈が$G$を要請している.
	\end{enumerate}
\end{quote}

\noindent(1) は\ref{firstargument}節で,(2) は\ref{secondargument}節で擁護する.

\section{表記法}
本論に入る前に,私が利用する表記法をまとめておく.まず文は,ギリシア文字$\phi, \psi, \ldots$で表す.そして\emph{説明}は,文の集合と文をつなぐ多対一のオペレーター$\xrightarrow{e}$として表現することにする(「e」はexplanationのeである).よって,$\phi_1, \phi_2, \ldots\xrightarrow{e}\psi$で,〈$\phi_1, \phi_2, \ldots$が$\psi$を説明する〉ということを意味する.

次に\emph{因果言明}は,やはり文の集合と文をつなぐ多対一のオペレーター$\xrightarrow{c}$として表現することにする(「c」はcausationのcである).よって,$\phi_1, \phi_2, \ldots\xrightarrow{c}\psi$で,〈$\phi_1, \phi_2, \ldots$が$\psi$を引き起こす〉ということを意味する.ただし,ここで因果言明は因果関係それ自体ではないことに注意せよ.あくまで因果言明は,しかじかの因果関係が成り立っていることを報告しているような言語的な表現である.よって,因果言明を文ベースで表すからといって,因果関係の関係項が文で支持されるような存在者(e.g., 事実,出来事)に限定されるということにはならない.

最後に\emph{グラウンディング言明}(グラウンディングの概念は\ref{firstargument}節で導入する)も,やはり文の集合と文をつなぐ多対一のオペレーター$\xrightarrow{g}$として表現することにする(「g」はgroundingのgである).よって,$\phi_1, \phi_2, \ldots\xrightarrow{g}\psi$で,〈$\phi_1, \phi_2, \ldots$が$\psi$をグラウンドする〉ということを意味する.上と同様に,この表現の仕方はグラウンディング関係の関係項についての特定のコミットメントを含意しない.

\section{制約による説明}\label{explanationbyconstraint}
\subsection{因果的説明}
\subsection{Langeの見解}

\section{「いかにして」の問いと「何のおかげで」の問い}\label{howandwhat}
説明の理論を立てるに際して,私たちは次の二つの問いを区別するべきである.
	\begin{description}
	\item[「いかにして」の問い:]$\phi\xrightarrow{e}\psi$について,\emph{いかにして}$\phi$は$\psi$を説明しているのか?
	\item[「何のおかげで」の問い:]$\phi\xrightarrow{e}\psi$について,\emph{何のおかげで}$\phi$は$\psi$を説明できているのか?
	\end{description}


\section{論証その1: 原因による説明とグラウンドによる説明}\label{firstargument}

\subsection{Langeの見解の難点その1}
Langeの理論には次の難点がある.
	\begin{enumerate}
	\item 必然性の階層構造は,形而上学的なコミットメントとして重すぎる.
	\item そもそも,「何のおかげで」の問いへの答えになっていない.
	\end{enumerate}

\subsection{グラウンディング}

\subsection{グラウンドによる説明としての制約による説明}
$\phi\xrightarrow{e}\psi$が非因果的説明であるのは,次のことが成り立つ\emph{おかげ}である.
	\begin{enumerate}
	\item $\phi$は$\psi$に関するグラウンディング情報$G$を含んでいる.
	\end{enumerate}

\section{論証その2: 文脈主義}\label{secondargument}

\subsection{Langeの見解の難点その2}
Langeの理論は,因果情報とグラウンディング情報は大抵の場合どちらも説明に必要だ,というデータを捉えられていない.

\subsection{発話の文脈}

\subsection{因果-グラウンディング混合理論の最終的な形}
$\phi\xrightarrow{e}\psi$が非因果的説明であるのは,次の二つが成り立つ\emph{おかげ}である.
	\begin{enumerate}
	\item $\phi$は$\psi$に関するグラウンディング情報$G$を含んでいる.	
	\item $\phi\xrightarrow{e}\psi$の発話の文脈が$G$を要請している.
	\end{enumerate}

\section{おわりに}

\section{メモ}
\cite{Lange2009a,Lange2009b,Lange2010,Lange2011,Lange2012,Lange2013a,Lange2013b,Lange2013c,Lange2014,Lange2015,Lange2016,Lange2018a,Lange2018b}

\subsection{\cite{Lange2011}. Conservation Laws in Scientific Explanations}
BWTのCh.2の前半部分の元ネタ(たぶん).

単なる偶然[coincidence]と制約[constraint]の違いの話をしている論文.特に,エネルギー保存則が,\emph{もし制約だったとしたら}それを使った説明の説明力の源泉は何か,を特定しようとしている.(それが実際に制約\emph{である},ということを示そうという論文ではない.)

%一般に,ある法則的言明には,\emph{ボトムアップの}説明と\emph{トップダウンの}説明が与えられうる.E.g., アルキメデスの原理には,力学的メカニズムに訴えるボトムアップの説明と,エネルギー保存則に訴えるトップダウンの説明が与えられる.

制約と偶然は,それぞれ次のように反事実によって定義される.
\begin{dfn}
エネルギー保存則が制約である\\
$\Longleftrightarrow$
現実にはない追加の力が存在する$\Box\hspace{-4pt}\rightarrow$エネルギーは保存される.\\
($A\Box\hspace{-4pt}\rightarrow C$)
\end{dfn}
\begin{dfn}
エネルギー保存則が偶然である\\
$\Longleftrightarrow$
現実にはない追加の力が存在する$\Diamond\hspace{-4pt}\rightarrow$エネルギーは保存されない.\\
($A\Diamond\hspace{-4pt}\rightarrow\lnot C\equiv \lnot(A\Box\hspace{-4pt}\rightarrow C)$)
\end{dfn}
制約の定義1で,反事実的条件法の前件に法則が入ってくることから,制約は\emph{メタ法則}だと言える\citep[170]{Lange2012}.

\subsection{\cite{Lange2012}. There Sweep}
BWTのCh.2の後半部分の元ネタ.

エネルギー保存則が制約なのか偶然なのかという区別は,何が何よりも説明的に優先するか[explanatorily prior]の区別\citep[157]{Lange2012}.
\begin{itemize}
	\item 制約であるなら,エネルギー保存則$<$個々の領域でのエネルギー保存.
	\item 偶然であるなら,個々の領域でのエネルギー保存$<$エネルギー保存則.
\end{itemize}

\emph{法則}の定義(暫定版)\citep[171--2]{Lange2012}:
\begin{dfn}
mが法則\\
$\Longleftrightarrow$ 全ての法則と論理的に整合的などんな命題pについても,($p\Box\hspace{-4pt}\rightarrow m$).
\end{dfn}


\emph{sub-nomic}と\emph{nomic}の区別\citep[171]{Lange2012}:
\begin{dfn}
pがsub-nomic\\
$\Longleftrightarrow$ pの真理メーカーが法則でない.
\end{dfn}
\begin{dfn}
pがnomic\\
$\Longleftrightarrow$ pの真理メーカーが法則.
\end{dfn}

\emph{sub-nomicな安定性}の定義\citep[172]{Lange2012}:
\begin{dfn}
$\Gamma$: sub-nomicな命題の集合で,$\Gamma=Cn\Gamma$.\\
$\Gamma$がsub-nomicな安定性を持つ\\
$\Longleftrightarrow(\forall m\in\Gamma)(\forall p)(\Gamma\cup\{p\}$は整合的$\rightarrow(\lnot(p\Diamond\hspace{-4pt}\rightarrow\lnot m))$.
\end{dfn}

\emph{法則}の定義(修正版)\citep[173]{Lange2012}:
\begin{dfn}
mが法則\\
$\Longleftrightarrow$ (極大でない)sub-nomicallyに安定した集合にmが属する.
\end{dfn}

\begin{thm}
sub-nomicallyに安定した集合のクラスは階層的である.\citep[173]{Lange2012}
\end{thm}

Langeの見解の利点\citep[174]{Lange2012}:
\begin{itemize}
	\item 様々なレベルの制約を認められる.
	\item 高階の法則としての制約の役割をうまく説明できる.(上の層のやつが下の層のやつを制約する.)
	\item 制約が種々の可能性を特徴づけるメカニズムの説明にもなってる.
\end{itemize}

\emph{nomicな安定性}の定義\citep[175]{Lange2012}:
\begin{dfn}
$\Gamma$: nomicないしsub-nomicな真理の集合で,$\Gamma=Cn\Gamma$.\\
$\Gamma$がnomicな安定性を持つ\\
$\Longleftrightarrow(\forall m\in\Gamma)(\forall p)(\Gamma\cup\{p\}は整合的\rightarrow (\lnot(p\Diamond\hspace{-4pt}\rightarrow\lnot m))$
\end{dfn}

最後の節で,本質主義的な法則観を退けている.(それは制約/偶然の区別をつけられないので.)

\subsection{\cite{Lange2013a}. What Makes a Scientific Explanation Distinctively Mathematical?}
BWCのCh.1の元ネタ.

非因果的説明を(ざっくり)定義している\citep[487,491]{Lange2013a}.
\begin{dfn}
    説明タイプEが非因果的である\\
    $\Longleftrightarrow$Eは,被説明項が生じるのがいかにして\emph{不可避}である(しかも,因果的必然性よりも強い意味で)のかを示すことによって説明している.
\end{dfn}
因果的説明も定義してる\citep[493]{Lange2013a}.
\begin{dfn}
    説明タイプEが因果的である\\
    $\Longleftrightarrow$ Eは,世界の因果情報を与えることによって説明している.
\end{dfn}


\subsection{\cite{Lange2018b}.Because Without Cause}
Langeの「制約による説明」についての概説的な論文.

ほげーー

\bibliographystyle{apa}
\bibliography{bibliography}

\end{document}  